\part{Architectures}
\section{Architectural styles}
\begin{slide}{Architectural styles}
  \begin{block}{Basic idea} 
    A style is formulated in terms of
    \begin{itemize}\tightlist
    \item (replaceable) components with well-defined interfaces
    \item the way that components are connected to each other
    \item the data exchanged between components
    \item how these components and connectors are jointly configured into a system.
    \end{itemize}
  \end{block}
  \begin{block}{Connector}
    A mechanism that mediates communication, coordination, or cooperation among components. \blue{Example}:
    facilities for (remote) procedure call, messaging, or streaming.
  \end{block}
\end{slide}
\subsection{Layered architectures}
\begin{slide}{Layered architecture}
  \begin{block}{Different layered organizations}
    \begin{center}
      \begin{tabular}{ccc}
        \includefigure{02-01a} &
        \includefigure{02-01b} &
        \includefigure{02-01c} \\
        (a) & (b) & (c) \\
      \end{tabular}
    \end{center}
  \end{block}
\end{slide}
\begin{slide}{Example: communication protocols}
  \begin{block}{Protocol, service, interface}
    \begin{center}
      \includefigure{02-02}
    \end{center}
  \end{block}
\end{slide}
  \begin{slide}{Two-party communication}
    \begin{block}{Server}
      \begin{tabular}{l}
        \includelisting{02-03/server-book}
      \end{tabular}
    \end{block}
    \begin{block}{Client}
      \begin{tabular}{l}
        \includelisting{02-03/client-book}
      \end{tabular}
    \end{block}
  \end{slide}
\begin{slide}{Application Layering}
  \begin{block}{Traditional three-layered view}
    \begin{itemize}\tightlist
    \item \blue{Application-interface layer} contains units for interfacing to users or external applications
    \item \blue{Processing layer} contains the functions of an application, i.e., without specific data
    \item \blue{Data layer} contains the data that a client wants to manipulate through the application
      components
    \end{itemize}
  \end{block}
  \onslide<2->
  \begin{alertblock}{Observation} 
    This layering is found in many distributed information systems, using traditional database technology and
    accompanying applications.
  \end{alertblock}
\end{slide}
\begin{slide}{Application Layering}
  \begin{block}{Example: a simple search engine}
    \begin{center}
      \includefigure{02-04}
    \end{center}
  \end{block}
\end{slide}
\subsection{Service-oriented architectures}
\begin{slide}{Object-based style}
  \begin{block}{Essence}
    Components are objects, connected to each other through procedure calls. Objects may be placed on
    different machines; calls can thus execute across a network.
  \end{block}
  \begin{block}{}
    \begin{tabular}{c@{\hspace*{24pt}}c}
      \includefigure{02-05a} &
      \includefigure{02-05b} \\
    \end{tabular}
  \end{block}
  \begin{block}{Encapsulation}
    Objects are said to \red{encapsulate data} and offer \red{methods on that data} without revealing the
    internal implementation.
  \end{block}
\end{slide}
\begin{slide}{RESTful architectures}
  \begin{block}{Essence}
    View a distributed system as a collection of resources, individually managed by components. Resources may
    be added, removed, retrieved, and modified by (remote) applications.
    \begin{enumerate}\tightlist
    \item Resources are identified through a single naming scheme
    \item All services offer the same interface
    \item Messages sent to or from a service are fully self-described
    \item After executing an operation at a service, that component forgets everything about the caller
    \end{enumerate}
  \end{block}
  \begin{block}{Basic operations}\scriptsize
    \begin{center}
      \begin{tabular}{|l|l|} \hline
        \textbf{Operation} & \textbf{Description} \\ \whline
        \code{PUT}    & Create a new resource \\ \hline
        \code{GET}    & Retrieve the state of a resource in some representation \\ \hline
        \code{DELETE} & Delete a resource \\ \hline
        \code{POST}   & Modify a resource by transferring a new state \\ \hline
      \end{tabular}
    \end{center}
  \end{block}
\end{slide}
\begin{slide}{Example: Amazon's Simple Storage Service}
  \begin{block}{Essence}
    \red{Objects} (i.e., files) are placed into \red{buckets} (i.e., directories). Buckets cannot be placed
    into buckets. Operations on \code{ObjectName} in bucket \code{BucketName} require the following
    identifier:
    \begin{center}
      \code{http://BucketName.s3.amazonaws.com/ObjectName}
    \end{center}
  \end{block}
  \begin{exampleblock}{Typical operations}
    All operations are carried out by sending HTTP requests:
    \begin{itemize}\tightlist
    \item Create a bucket/object: \code{PUT}, along with the URI
    \item Listing objects: \code{GET} on a bucket name
    \item Reading an object: \code{GET} on a full URI
    \end{itemize}
  \end{exampleblock}
\end{slide}
  \begin{slide}{On interfaces}
    \begin{alertblock}{Issue}
      Many people like RESTful approaches because the interface to a service is so simple. The catch is that
      much needs to be done in the \blue{parameter space}.
    \end{alertblock}
    \begin{block}{Amazon S3 SOAP interface}
      \begin{centerfig}
        \includefigure{02-08}
      \end{centerfig}
    \end{block}
  \end{slide}
  \begin{slide}{On interfaces}
    \begin{block}{Simplifications}
      Assume an interface \code{bucket} offering an operation \code{create}, requiring an
      input string such as \code{mybucket}, for creating a bucket ``mybucket.''
    \end{block}
    \onslide<2->
    \begin{exampleblock}{SOAP}
      \begin{quote}
        \code{import bucket} \newline
        \code{bucket.create("mybucket")}
      \end{quote}
    \end{exampleblock}
    \onslide<3->
    \begin{exampleblock}{RESTful}
      \begin{quote}
        \code{PUT "https://mybucket.s3.amazonsws.com/"}
      \end{quote}
    \end{exampleblock}
    \onslide<4->
    \begin{block}{Conclusions}
      Are there any to draw?
    \end{block}
  \end{slide}
\subsection{Publish-subscribe architectures}
\begin{slide}{Coordination}
  \begin{block}{Temporal and referential coupling}
    \begin{center}
      \sffamily\footnotesize
      \renewcommand{\arraystretch}{1}
      \begin{tabular}{|rIc|c|} \hline
                                  & \textbf{Temporally coupled} & \textbf{Temporally coupled} \\ \whline
        \textbf{Referentially coupled}    & Direct              & Mailbox     \\ \hline
        \textbf{Referentially decoupled}    & Event-based              & Shared data space      \\ \hline
      \end{tabular}
    \end{center}
  \end{block}
  \begin{block}{Event-based and Shared data space}
    \begin{center}
      \begin{tabular}{@{}cc}
        \vtop{\null\hbox{\includefigure{02-10a}}} &
        \vtop{\null\hbox{\includefigure{02-10b}}}
      \end{tabular}
    \end{center}
  \end{block}
\end{slide}
  \begin{slide}{Example: Linda tuple space}
    \begin{block}{Three simple operations}
      \begin{itemize}\tightlist
      \item \codesn{in(t)}: remove a tuple matching template \codesn{t}
      \item \codesn{rd(t)}: obtain copy of a tuple matching template \codesn{t}
      \item \codesn{out(t)}: add tuple \codesn{t} to the tuple space
      \end{itemize}
    \end{block}
    \begin{block}{More details}
      \begin{itemize}\tightlist
      \item Calling \codesn{out(t)} twice in a row, leads to storing \red{two} copies of tuple \codesn{t}
        \mathexpr{\Rightarrow} a tuple space is modeled as a \blue{multiset}.
      \item Both \codesn{in} and \codesn{rd} are \blue{blocking} operations: the caller will be blocked until
        a matching tuple is found, or has become available.
      \end{itemize}
    \end{block}
  \end{slide}
  \begin{slide}{Example: Linda tuple space}
    \begin{tabular}{l@{\hspace*{1cm}}l}
      \blue{Bob}:   & \includelisting{02-11/bob}   \\ \hline
      \blue{Alice}: & \includelisting{02-11/alice} \\ \hline
      \blue{Chuck}: & \includelisting{02-11/chuck} \\ 
    \end{tabular}
  \end{slide}
\begin{slide}{Publish and subscribe}
  \begin{block}{Issue: how to match events?}
    \begin{itemize}\firmlist
    \item Assume events are described by \blue{(attribute,value)} pairs
    \item \red{topic-based subscription}: specify a ``\blue{attribute = value}'' series
    \item \red{content-based subscription}: specify a ``\blue{attribute \mathexpr{\in} range}'' series
    \end{itemize}
  \end{block}
  \centering\includefigure{02-12}
  \begin{alertblock}{Observation}
    Content-based subscriptions may easily have serious scalability problems (\blue{why?})
  \end{alertblock}
\end{slide}
\section{Middleware and distributed systems}
\begin{slide}{Middleware: the OS of distributed systems}
  \centering\includefigure{02-13}
  \begin{block}{What does it contain?}
    Commonly used components and functions that need not be implemented by applications separately.
  \end{block}
\end{slide}
\subsection{Middleware organization}
\begin{slide}{Using legacy to build middleware}
  \begin{block}{Problem}
    The interfaces offered by a legacy component are most likely not suitable for all applications.
  \end{block}
  \begin{block}{Solution}
    A \red{wrapper} or \red{adapter} offers an interface acceptable to a client application. Its functions are
    transformed into those available at the component.
  \end{block}
\end{slide}
\begin{slide}{Organizing wrappers}
  \begin{block}{Two solutions: 1-on-1 or through a broker}
    \begin{center}
      \begin{tabular}{c@{\hspace*{2cm}}c}
        \includefigure{02-14a} &
        \includefigure{02-14b} \\
      \end{tabular}
    \end{center}
  \end{block}
  \begin{block}{Complexity with \mathexpr{N} applications}
    \begin{itemize}
    \item \blue{1-on-1}: requires \mathexpr{N \times (N-1) = \mathcal{O}(N^2)} wrappers
    \item \blue{broker}: requires \mathexpr{2N = \mathcal{O}(N)} wrappers 
    \end{itemize}
  \end{block}
\end{slide}
\begin{slide}{Developing adaptable middleware}
  \begin{block}{Problem} 
    Middleware contains solutions that are good for \blue{most} applications \mathexpr{\Rightarrow} you may
    want to adapt its behavior for specific applications.
  \end{block}
\end{slide}
\begin{slide}{Intercept the usual flow of control}
  \begin{block}{}
    \begin{center}
        \includefigure{02-15}
    \end{center}
  \end{block}
\end{slide}
\subsection{Modifiable middleware}
\section{Layered-system architectures}
\subsection{Simple client-server architecture}
\begin{slide}{Centralized system architectures}
  \begin{block}{Basic Client--Server Model}
    Characteristics:
    \begin{itemize}\tightlist
    \item There are processes offering services (\red{servers}) 
    \item There are processes that use services (\red{clients})
    \item Clients and servers can be on different machines
    \item Clients follow request/reply model regarding using services
    \end{itemize}
    \begin{center}
      \includefigure{02-16}
    \end{center}
  \end{block}
\end{slide}
\subsection{Multitiered Architectures}
\begin{slide}{Multi-tiered centralized system architectures}
  \begin{block}{Some traditional organizations}
    \begin{itemize}\tightlist
    \item \red{Single-tiered:} dumb terminal/mainframe configuration
    \item \red{Two-tiered:} client/single server configuration
    \item \red{Three-tiered:} each layer on separate machine
    \end{itemize}
  \end{block}
  \begin{block}{Traditional two-tiered configurations}
    \newcommand{\ispace}{\hspace*{1.8cm}}
    \begin{center}
      \includefigure{02-17} \newline
      (a)\ispace(b)\ispace(c)\ispace(d)\ispace(e)
    \end{center}
  \end{block}
\end{slide}
\begin{slide}{Being client and server at the same time}
  \begin{block}{Three-tiered architecture}
    \vspace*{12pt}
    \centering\includefigure{02-18}
  \end{block}
\end{slide}
\subsection{Example: The Network File System}
\begin{slide}{Example: The Network File System}
  \begin{block}{Foundations}
    Each NFS server provides a standardized view of its local file system: each server supports the same
    model, regardless the implementation of the file system.
  \end{block}
  \begin{block}{The NFS remote access model}
    \vspace*{-6pt}
    \begin{centerfig}
      \begin{tabular}{@{}cc@{}}
        \includefigure{02-19a} &
        \includefigure{02-19b} \\
        \vspace*{-6pt}
        Remote access & Upload/download 
      \end{tabular}
    \end{centerfig}
    \vspace*{-12pt}
  \end{block}
  \begin{exampleblock}{Note}
    FTP is a typical upload/download model. The same can be said for systems like Dropbox.
  \end{exampleblock}
\end{slide}
\begin{slide}{NFS architecture}
  \centering\includefigure{02-20}
\end{slide}
\subsection{Example: The Web}
\begin{slide}{Example: Simple Web servers}
  \begin{block}{Back in the old days...}
    \centering\includefigure{02-21}
  \end{block}
  \begin{block}{...life was simple:}
    \begin{itemize}\firmlist
    \item A website consisted as a collection of HTML files
    \item HTML files could be referred to each other by a \blue{hyperlink}
    \item A Web server essentially needed only a hyperlink to fetch a file
    \item A browser took care of properly rendering the content of a file
    \end{itemize}
  \end{block}
\end{slide}
\begin{slide}{Example (cnt'd): Less simple Web servers}
  \begin{block}{Still back in the old days...}
    \centering\includefigure{02-22}
  \end{block}
  \begin{block}{...life became a bit more complicated:}
    \begin{itemize}\firmlist
    \item A website was built around a database with content
    \item A Webpage could still be referred to by a \blue{hyperlink}
    \item A Web server essentially needed only a hyperlink to fetch a file
    \item A separate program (\red{Common Gateway Interface}) \blue{composed} a page
    \item A browser took care of properly rendering the content of a file
    \end{itemize}
  \end{block}

\end{slide}
\section{Symmetrically distributed system architectures}
\begin{slide}{Alternative organizations}
  \begin{block}{Vertical distribution}
    Comes from dividing distributed applications into three logical layers, and running the components
    from each layer on a different server (machine).
  \end{block}
  \begin{block}{Horizontal distribution}
    A client or server may be physically split up into logically equivalent parts, but each part is operating
    on its own share of the complete data set.
  \end{block}
  \begin{block}{Peer-to-peer architectures}
    Processes are all equal: the functions that need to be carried out are represented by every process
    \mathexpr{\Rightarrow} each process will act as a client and a server at the same time (i.e.,
    acting as a \red{servant}).
  \end{block}
\end{slide}
\subsection{Structured peer-to-peer systems}
\begin{slide}{Structured P2P}
  \begin{block}{Essence}
    Make use of a \blue{semantic-free index}: each data item is uniquely associated with a key, in turn used
    as an index. Common practice: use a \red{hash function}
    \vspace*{-6pt}
    \[ \func{key}(\func{data item}) = \func{hash}(\func{data item's value}). \vspace*{-6pt}\]
    P2P system now responsible for storing (\emph{key},\emph{value}) pairs. 
  \end{block}
  \begin{exampleblock}{Simple example: hypercube}
    \begin{centerfig}
      \includefigure{02-23}
    \end{centerfig}
    Looking up \mathexpr{d} with \red{key} \mathexpr{k \in \{0,1,2,\ldots,2^4-1\}} means \blue{routing}
    request to node with \red{identifier} \mathexpr{k}.
  \end{exampleblock}
\end{slide}
  \begin{slide}{Example: Chord}
    \begin{block}{Principle}
      \begin{itemize}\tightlist
      \item Nodes are logically organized in a ring. Each node has an \mathexpr{m}-bit \red{identifier}.
      \item Each data item is hashed to an \mathexpr{m}-bit \red{key}.
      \item Data item with key \mathexpr{k} is stored at node with smallest identifier \mathexpr{\id{id} \geq
        k}, called the \red{successor} of key \mathexpr{k}.
      \item The ring is extended with various \blue{shortcut links} to other nodes.
      \end{itemize}
    \end{block}
  \end{slide}
  \begin{slide}{Example: Chord}
    \begin{centerfig}
      \begin{tabular}{c}
        \includefigure{02-24}\\
        \mathexpr{\func{lookup}(3)@9: 28 \rightarrow 1 \rightarrow 4} 
      \end{tabular}
    \end{centerfig}
  \end{slide}
\subsection{Unstructured peer-to-peer systems}
\begin{slide}{Unstructured P2P}
  \begin{block}{Essence}
    Each node maintains an ad hoc list of neighbors. The resulting overlay resembles a \red{random graph}: an
    edge \mathexpr{\langle \id{u},\id{v} \rangle} exists only with a
    certain probability \mathexpr{\mathbb{P}[\langle \id{u}, \id{v} \rangle]}.
  \end{block}
  \begin{block}{Searching}
    \begin{itemize}
    \item \red{Flooding}: issuing node \id{u} passes request for \mathexpr{d} to all neighbors. Request is
      ignored when receiving node had seen it before. Otherwise, \id{v} searches locally for \mathexpr{d}
      (recursively). May be limited by a \red{Time-To-Live}: a maximum number of hops.
    \item \red{Random walk}: issuing node \id{u} passes request for \mathexpr{d} to randomly chosen neighbor,
      \id{v}. If \id{v} does not have \mathexpr{d}, it forwards request to one of \emph{its} randomly chosen
      neighbors, and so on.
    \end{itemize}
  \end{block}
\end{slide}
  \begin{slide}{Flooding versus random walk}
    \begin{block}{Model}
      Assume \mathexpr{N} nodes and that each data item is replicated across \mathexpr{r} randomly chosen
      nodes.
    \end{block}
    \begin{block}{Random walk}
      \mathexpr{\mathbb{P}[k]} probability that item is found after \mathexpr{k} attempts:
      \[ \mathbb{P}[k] = \frac{r}{N}(1-\frac{r}{N})^{k-1}.\]
      \mathexpr{S} (``search size'') is expected number of nodes that need to be probed:
      \[ S = \sum_{k=1}^N k \cdot \mathbb{P}[k] = \sum_{k=1}^N k \cdot \frac{r}{N}(1-\frac{r}{N})^{k-1}
      \approx N/r \mbox{\ for \mathexpr{1 \ll r \leq N}}. \]
    \end{block}
  \end{slide}
  \begin{slide}{Flooding versus random walk}
    \begin{block}{Flooding}
      \begin{itemize}\tightlist
      \item Flood to \mathexpr{d} randomly chosen neighbors
      \item After \mathexpr{k} steps, some \mathexpr{R(k) = d \cdot (d-1)^{k-1}} will have been reached (assuming
        \mathexpr{k} is small).
      \item With fraction \mathexpr{r/N} nodes having data, if \mathexpr{\frac{r}{N}\cdot R(k) \geq 1}, we
        will have found the data item.
      \end{itemize}
    \end{block}
    \begin{exampleblock}{Comparison}
      \begin{itemize}
      \item If \mathexpr{r/N = 0.001}, then \mathexpr{S \approx 1000}
      \item With flooding and \mathexpr{d = 10, k = 4}, we contact 7290 nodes.
      \item Random walks are more communication efficient, but might take longer before they find the result.
      \end{itemize}
    \end{exampleblock}
  \end{slide}
\subsection{Hierarchically organized peer-to-peer networks}
\begin{slide}{Super-peer networks}
  \begin{block}{Essence}
    It is sometimes sensible to break the symmetry in pure peer-to-peer networks:
    \begin{itemize}\tightlist
    \item When searching in unstructured P2P systems, having \red{index servers} improves performance
    \item Deciding where to store data can often be done more efficiently through \red{brokers}.
    \end{itemize}
  \end{block}
  \begin{centerfig}
    \includefigure{02-25}
  \end{centerfig}
\end{slide}
\subsection{Example: BitTorrent}
\begin{slide}{Collaboration: The BitTorrent case}
  \begin{block}{Principle: search for a file \id{F}}
    \begin{itemize}\tightlist
    \item Lookup file at a global directory \mathexpr{\Rightarrow} returns a \red{torrent file}
    \item Torrent file contains reference to \red{tracker}: a server keeping an accurate account of
      \blue{active} nodes that have (chunks of) \id{F}.
    \item \id{P} can join \red{swarm}, get a chunk for free, and then trade a copy of that chunk for another
      one with a peer \id{Q} also in the swarm.
    \end{itemize}
  \end{block}
  \begin{centerfig}
    \includefigure{02-26}
  \end{centerfig}
\end{slide}
\section{Hybrid system architectures}
\subsection{Cloud computing}
\begin{slide}{Cloud computing}
  \begin{centerfig}
    \includefigure{02-27}
  \end{centerfig}
\end{slide}
\begin{slide}{Cloud computing}
  \begin{block}{Make a distinction between four layers}
    \begin{itemize}
    \item \red{Hardware}: Processors, routers, power and cooling systems. Customers normally never get to see these.
    \item \red{Infrastructure}: Deploys virtualization techniques. Evolves around allocating and managing
      virtual storage devices and virtual servers.
    \item \red{Platform}: Provides higher-level abstractions for storage and such. Example: Amazon S3 storage
      system offers an API for (locally created) files to be organized and stored in so-called \blue{buckets}.
    \item \red{Application}: Actual applications, such as office suites (text processors, spreadsheet
      applications, presentation applications). Comparable to the suite of apps shipped with OSes.
    \end{itemize}
  \end{block}
\end{slide}
\subsection{The edge-cloud architecture}
\begin{slide}{Edge-server architecture}
  \begin{block}{Essence}
    Systems deployed on the Internet where servers are placed \blue{at the edge} of the network:
    the boundary between enterprise networks and the actual Internet.
  \end{block}
  \begin{centerfig}
    \includefigure{02-28}
  \end{centerfig}
\end{slide}
\begin{slide}{Reasons for having an edge infrastructure}
  \begin{block}{Commonly (and often misconceived) arguments}
    \begin{itemize}
    \item \blue{Latency and bandwidth}: Especially important for certain real-time applications, such as
      augmented/virtual reality applications. Many people underestimate the latency and bandwidth to the
      cloud.
    \item \blue{Reliability}: The connection to the cloud is often assumed to be unreliable, which is often a
      false assumption. There may be critical situations in which extremely high connectivity guarantees are
      needed.
    \item \blue{Security and privacy}: The implicit assumption is often that when assets are nearby, they can
      be made better protected. Practice shows that this assumption is generally false. However, securely
      handling data operations in the cloud may be trickier than within your own organization.
    \end{itemize}
  \end{block}
\end{slide}
\begin{slide}{Edge orchestration}
  \begin{block}{Managing resources at the edge may be trickier than in the cloud}
    \begin{itemize}\firmlist

    \item \red{Resource allocation}: we need to guarantee the availability of the resources required to
      perform a service. 

    \item \red{Service placement}: we need to decide \blue{when} and \blue{where} to place a service. This is
      notably relevant for mobile applications.

    \item \red{Edge selection}: we need to decide which edge infrastructure should be
      used when a service needs to be offered. The closest one may not be the best one.
    \end{itemize}
  \end{block}

  \begin{alertblock}{Observation}
    There is still a lot of buzz about edge infrastructures and computing, yet whether all that buzz makes any
    sense remains to be seen.
  \end{alertblock}
\end{slide}
\subsection{Blockchain architectures}
\begin{slide}{Blockchains}
  \begin{block}{Principle working of a blockchain system}
    \centering\includefigure{02-29}
  \end{block}
  \onslide<2->
  \begin{block}{Observations}
    \begin{itemize}\firmlist
    \item Blocks are organized into an unforgeable \blue{append-only} chain
    \item Each block in the blockchain is \blue{immutable} \mathexpr{\Rightarrow} massive replication
    \item The real snag lies in who is allowed to append a block to a chain
    \end{itemize}
    
  \end{block}
\end{slide}
\begin{slide}{Appending a block: distributed consensus}
  \begin{block}{Centralized solution}
    \begin{centerfig}
      \includefigure{02-30a}
    \end{centerfig}
  \end{block}
  \begin{block}{Observation}
    A single entity decides on which validator can go ahead and append a block. Does not fit the design goals
    of blockchains.
  \end{block}
\end{slide}
\begin{slide}{Appending a block: distributed consensus}
  \begin{block}{Distributed solution (permissioned)}
    \begin{centerfig}
      \includefigure{02-30b}
    \end{centerfig}
  \end{block}
  \begin{block}{Observation}
    \begin{itemize}\firmlist
    \item A selected, relatively small group of servers jointly reach consensus on which validator can go
      ahead.
    \item None of these servers needs to be trusted, as long as roughly two-thirds behave according to their
      specifications.
    \item In practice, only a few tens of servers can be accommodated.
    \end{itemize}
  \end{block}
\end{slide}
\begin{slide}{Appending a block: distributed consensus}
  \begin{block}{Decentralized solution (permisionless)}
    \begin{centerfig}
      \includefigure{02-30c}
    \end{centerfig}
  \end{block}
  \begin{block}{Observation}
    \begin{itemize}\firmlist
    \item Participants collectively engage in a \red{leader election}. Only the elected leader is allowed to
      append a block of validated transactions.
    \item Large-scale, decentralized leader election that is fair, robust, secure, and so on, is far from
      trivial. 
    \end{itemize}
    
  \end{block}
\end{slide}
\section{Summary}
