\part{Consistency and replication}
\section{Introduction}
\subsection{Reasons for replication}
\begin{slide}{Replication}
  \begin{block}{Why replicate}
    Assume a simple model in which we make a copy of a specific part of a system (meaning code and data).
    \begin{itemize}
    \item \red{Increase reliability}: if one copy does not live up to specifications, switch over to the other
      copy while repairing the failing one.
    \item \red{Performance}: simply spread requests between different replicated parts to keep load balanced,
      or to ensure quick responses by taking proximity into account.
    \end{itemize}
    \begin{alertblock}{The problem}
      Having multiple copies, means that when any copy changes, that change should be made at all copies:
      \blue{replicas need to be kept the same}, that is, be kept \red{consistent}.
    \end{alertblock}
  \end{block}
  
\end{slide}
\begin{slide}{Performance and scalability}
  \begin{block}{Main issue} 
    To keep replicas consistent, we generally need to ensure that all \blue{conflicting} operations are done
    in the the same order everywhere
  \end{block}
  \begin{block}{Conflicting operations: From the world of transactions}
    \begin{itemize}\tightlist
    \item \blue{Read--write conflict}: a read operation and a write operation act concurrently
    \item \blue{Write--write conflict}: two concurrent write operations
    \end{itemize}
  \end{block}
  \begin{block}{Issue}
    Guaranteeing global ordering on conflicting operations may be a costly operation, downgrading scalability.
    \red{Solution:} weaken consistency requirements so that hopefully global synchronization can be avoided
  \end{block}
\end{slide}
\subsection{Replication as scaling technique}
\section{Data-centric consistency models}
\begin{slide}{Data-centric consistency models}
  \begin{block}{Consistency model} 
    A contract between a (distributed) data store and processes, in which the data store specifies precisely
    what the results of read and write operations are in the presence of concurrency.
  \end{block}
  \begin{alertblock}{Essential}
    A data store is a distributed collection of storages:
    \begin{center}
      \includefigure{07-01}      
    \end{center}
  \end{alertblock}
\end{slide}
\subsection{Consistent ordering of operations}
\begin{slide}{Some notations}
  \begin{block}{Read and write operations}
    \begin{itemize}
    \item \red{\id{W_i(x)a}}: Process \id{P_i} writes value \id{a} to \id{x}
    \item \red{\id{R_i(x)b}}: Process \id{P_i} reads value \id{b} from \id{x}
    \item All data items initially have value \red{\id{NIL}}
    \end{itemize}
  \end{block}

  \begin{block}{Possible behavior}
    We omit the index when possible and draw according to time (x-axis):

    \vspace*{12pt}
    \begin{centerfig}
      \includefigure[1.2]{07-02}
    \end{centerfig}
  \end{block}
\end{slide}
\begin{slide}{Sequential consistency}
  \begin{block}{Definition}
    The result of any execution is the same as if the operations of all processes were executed in some
    sequential order, and the operations of each individual process appear in this sequence in the order
    specified by its program.
  \end{block}
  \vspace*{-6pt}
  \begin{centerfig}
    \includefigure[1]{07-03a} \\
    A sequentially consistent data store \\
    \vspace*{12pt}
    \includefigure[1]{07-03b} \\
    A data store that is not sequentially consistent
  \end{centerfig}
\end{slide}
\begin{slide}{Example}
  \begin{block}{Three concurrent processes (initial values: 0)}
    \begin{center}
      \renewcommand{\arraystretch}{0.9}
      \begin{tabular}{lll}
        Process \id{P_1} & Process \id{P_2} & Process \id{P_3} \\ \whline 
        \code{x \mathexpr{\leftarrow} 1;} & \code{y \mathexpr{\leftarrow} 1;} & \code{z \mathexpr{\leftarrow} 1;} \\
        \code{print(y,z);}                & \code{print(x,z);}                & \code{print(x,y);}
      \end{tabular}
    \end{center}
  \end{block}
  \onslide
  \begin{block}{Example execution sequences}
    \begin{centerfig}
      \includefigure{07-05}
    \end{centerfig}
  \end{block}
\end{slide}
  \begin{slide}{How tricky can it  get?}
    \begin{block}{Seemingly okay}
      \centering\includefigure[1.2]{07-06}
    \end{block}
    \onslide
    \begin{block}{But not really (don't forget that \id{P_1} and \id{P_2} act concurrently)}
      \begin{center}
        \begin{tabular}{|c@{;\ }c@{;\ }c@{;\ }c|c|c|}\hline
          \multicolumn{4}{|c|}{\textbf{Possible ordering of operations}} & \multicolumn{2}{c|}{\textbf{Result}} \\ \whline
          \idsn{W_1(x)a} & \idsn{W_1(y)a} & \idsn{W_2(y)b} & \idsn{W_2(x)b} & \idsn{R_1(x)b} & \idsn{R_2(y)b} \\
          \idsn{W_1(x)a} & \idsn{W_2(y)b} & \idsn{W_1(y)a} & \idsn{W_2(x)b} & \idsn{R_1(x)b} & \idsn{R_2(y)a} \\
          \idsn{W_1(x)a} & \idsn{W_2(y)b} & \idsn{W_2(x)b} & \idsn{W_1(y)a} & \idsn{R_1(x)b} & \idsn{R_2(y)a} \\
          \idsn{W_2(y)b} & \idsn{W_1(x)a} & \idsn{W_1(y)a} & \idsn{W_2(x)b} & \idsn{R_1(x)b} & \idsn{R_2(y)a} \\
          \idsn{W_2(y)b} & \idsn{W_1(x)a} & \idsn{W_2(x)b} & \idsn{W_1(y)a} & \idsn{R_1(x)b} & \idsn{R_2(y)a} \\
          \idsn{W_2(y)b} & \idsn{W_2(x)b} & \idsn{W_1(x)a} & \idsn{W_1(y)a} & \idsn{R_1(x)a} & \idsn{R_2(y)a} \\ \hline
        \end{tabular}
      \end{center}
    \end{block}
  \end{slide}
  \begin{slide}{How tricky can it get?}
    \begin{block}{Linearizability}
      Each operation should appear to take effect instantaneously at some moment between its start and completion.
    \end{block}
    \begin{block}{Operations complete within a given time (shaded area)}
      \centering\includefigure[1.2]{07-07}
    \end{block}
    \begin{block}{With better results}
      \begin{center}
        \begin{tabular}{|c@{;\ }c@{;\ }c@{;\ }c|c|c|}\hline
          \multicolumn{4}{|c|}{\textbf{Possible ordering of operations}} & \multicolumn{2}{c|}{\textbf{Result}} \\ \whline
          \idsn{W_1(x)a} & \idsn{W_2(y)b} & \idsn{W_1(y)a} & \idsn{W_2(x)b} & \idsn{R_1(x)b} & \idsn{R_2(y)a} \\
          \idsn{W_1(x)a} & \idsn{W_2(y)b} & \idsn{W_2(x)b} & \idsn{W_1(y)a} & \idsn{R_1(x)b} & \idsn{R_2(y)a} \\
          \idsn{W_2(y)b} & \idsn{W_1(x)a} & \idsn{W_1(y)a} & \idsn{W_2(x)b} & \idsn{R_1(x)b} & \idsn{R_2(y)a} \\
          \idsn{W_2(y)b} & \idsn{W_1(x)a} & \idsn{W_2(x)b} & \idsn{W_1(y)a} & \idsn{R_1(x)b} & \idsn{R_2(y)a} \\ \hline
        \end{tabular}
      \end{center}
    \end{block}
  \end{slide}
\begin{slide}{Causal consistency}
  \begin{block}{Definition}
    Writes that are potentially causally related must be seen by all processes in the same order. Concurrent
    writes may be seen in a different order by different processes.
  \end{block}
  \vspace*{-6pt}
  \begin{centerfig}
    \includefigure[1]{07-09a} \\
    A violation of a causally-consistent store \\
    \vspace*{12pt}
    \includefigure[1]{07-09b} \\
    A correct sequence of events in a causally-consistent store
  \end{centerfig}
\end{slide}
  \begin{slide}{Consistency models, serializability, transactions}
    \begin{block}{Overwhelming, but often already known}
      Again, from the world of \blue{transactions}: can we order the execution of all operations in a set of
      transactions in such a way that the final result matches a serial execution of those transactions? The
      keyword is \red{serializability}.
    \end{block}
    \begin{centerfig}
      \newcommand{\si}{\hspace*{1em}}
      \renewcommand{\arraystretch}{0.9}
      \begin{tabular}{l@{\hspace*{1cm}}l@{\hspace*{1cm}}l}
        \code{BEGIN\_TRANSACTION} & \code{BEGIN\_TRANSACTION} & \code{BEGIN\_TRANSACTION} \\
        \si\code{x = 0}           & \si\code{x = 0}           & \si\code{x = 0}           \\
        \si\code{x = x + 1}       & \si\code{x = x + 2}       & \si\code{x = x + 3}       \\
        \code{END\_TRANSACTION}   & \code{END\_TRANSACTION}   & \code{END\_TRANSACTION}   \\ 
        \blue{Transaction \id{T_1}} & \blue{Transaction \id{T_2}} & \blue{Transaction \id{T_3}} \\
      \end{tabular}
    \end{centerfig}
    \begin{exampleblock}{A number of schedules}
      \begin{centerfig}
        \sffamily\footnotesize
        \begin{tabular}{|l|llllll|l|}\hline
          & \multicolumn{6}{c|}{Time \mathexpr{\longrightarrow}} & \\ \hline
          S1 & x = 0 & x = x + 1 & x = 0     & x = x + 2 & x = 0     & x = x + 3 & \blue{Legal}   \\
          S2 & x = 0 & x = 0     & x = x + 1 & x = x + 2 & x = 0     & x = x + 3 & \blue{Legal}   \\
          S3 & x = 0 & x = 0     & x = x + 1 & x = 0     & x = x + 2 & x = x + 3 & \red{Illegal} \\
          S4 & x = 0 & x = 0     & x = x + 3 & x = 0     & x = x + 1 & x = x + 2 & \red{Illegal} \\ \hline
        \end{tabular}
      \end{centerfig}
    \end{exampleblock}
  \end{slide}
\begin{slide}{Grouping operations}
  \begin{block}{Entry consistency: Definition}
    \begin{itemize}
    \item Accesses to \red{locks} are sequentially consistent.
    \item No access to a lock is allowed to be performed until all previous writes have completed everywhere.
    \item No data access is allowed to be performed until all previous accesses to locks have been performed.
    \end{itemize}
  \end{block}
  \onslide
  \begin{block}{Basic idea} 
    You don't care that reads and writes of a \blue{series} of operations are immediately known to other
    processes. You just want the \blue{effect} of the series itself to be known.
  \end{block}
\end{slide}
\begin{slide}{Grouping operations}
  \begin{block}{A valid event sequence for entry consistency}
    \begin{center}
      \includefigure[1]{07-11}
    \end{center}
  \end{block}
  \begin{block}{Observation} 
    Entry consistency implies that we need to lock and unlock data (implicitly or not).
  \end{block}
  \begin{block}{Question}
    What would be a convenient way of making this consistency more or less transparent to programmers?
  \end{block}
\end{slide}
\subsection{Eventual consistency}
\begin{slide}{Eventual consistency}
  \begin{block}{Definition}
    Consider a collection of data stores and (concurrent) write operations. The strores are \blue{eventually
      consistent} when in lack of updates from a certain moment, all updates to that point are propagated in
    such a way that replicas will have the same data stored (until updates are accepted again).
  \end{block}

  \begin{block}{Srong eventual consistency}
    Basic idea: if there are conflicting updates, have a globally determined resolution mechanism (for
    example, using NTP, simply let the ``most recent'' update win). 
  \end{block}

  \begin{block}{Program consistency}
    \mathexpr{P} is a \red{monotonic problem} if for any input sets \mathexpr{S} and \mathexpr{T},
    \mathexpr{P(S) \subseteq P(T)}.

    \red{Observation}: A \blue{program} solving a monotonic problem can start with incomplete information, but
    is guaranteed not to have to roll back when missing information becomes available. \green{Example}:
    filling a shopping cart.
    NOTE: \emph{In all cases, we are avoiding global 
    synchronization.}
  \end{block}
\end{slide}
\subsection{Continuous consistency}
\begin{slide}{Continuous Consistency}
  \begin{block}{We can actually talk about a \blue{degree of consistency}}
    \begin{itemize}\tightlist
    \item replicas may differ in their \blue{numerical value}
    \item replicas may differ in their relative \blue{staleness}
    \item there may be differences regarding (number and order) of \blue{performed update operations}
    \end{itemize}
  \end{block}
  \begin{alertblock}{Conit}
    Consistency unit \mathexpr{\Rightarrow} specifies the \blue{data unit} over which consistency is to be measured.
  \end{alertblock}
\end{slide}
\begin{slide}{Example: Conit}
  \begin{center}
    \includefigure{07-15}
  \end{center}
  \begin{block}{Conit (contains the variables \id{g}, \id{p}, and \id{d})}
    \begin{itemize}\tightlist
    \item Each replica has a \blue{vector clock}: 
      \red{([known] time @ A, [known] time @ B)}
    \item \id{B} sends \id{A} operation [\mathexpr{\langle 5, \id{B} \rangle: \id{g \leftarrow d + 45}}];
      \mathexpr{A} has made this operation \blue{permanent} (cannot be rolled back)
    \end{itemize}
  \end{block}
\end{slide}
\begin{slide}{Example: Conit}
  \begin{center}
    \includefigure{07-15}
  \end{center}
  \begin{block}{Conit (contains the variables \id{g}, \id{p}, and \id{d})}
    \begin{itemize}\tightlist
    \item \id{A} has three \blue{pending} operations \mathexpr{\Rightarrow} order deviation = 3
    \item \id{A} missed \blue{two} operations from \id{B}; max diff is 70 + 412 units
      \mathexpr{\Rightarrow} \mathexpr{(2,482)} 
    \end{itemize}
  \end{block}
\end{slide}
\section{Client-centric consistency models}
\begin{slide}{Consistency for mobile users}
  \begin{exampleblock}{Example}
    Consider a distributed database to which you have access through your notebook. Assume your notebook acts
    as a front end to the database.
    \begin{itemize}
    \item At location \id{A} you access the database doing reads and updates.
    \item At location \id{B} you continue your work, but unless you access the same server as the one at location
      \mathexpr{A}, you may detect inconsistencies:
      \begin{itemize}\tightlist
      \item your updates at \id{A} may not have yet been propagated to \id{B}
      \item you may be reading newer entries than the ones available at \id{A}
      \item your updates at \id{B} may eventually conflict with those at \id{A}
      \end{itemize}
    \end{itemize}
  \end{exampleblock}
  \onslide
  \begin{alertblock}{Note}
    The only thing you really want is that the entries you updated and/or read at \id{A}, are in \id{B} the
    way you left them in \id{A}. In that case, the database will appear to be consistent \blue{to you}.
  \end{alertblock}
\end{slide}
\begin{slide}{Basic architecture}
  \begin{block}{The principle of a mobile user accessing different replicas of a distributed database}
    \begin{centerfig}
      \includefigure{07-17}
    \end{centerfig}
  \end{block}
\end{slide}
\subsection{Monotonic reads}
\begin{slide}{Client-centric consistency: notation}
  \begin{block}{Notations} 
    \begin{itemize}
    \item \red{\id{W_1(x_2)}} is the write operation by process \id{P_1} that leads to \blue{version} \id{x_2} of \id{x}
    \item \red{\id{W_1(x_i;x_j)}} indicates \id{P_1} produces version \id{x_j} based on a previous version \id{x_i}.
    \item \red{\id{W_1(x_i|x_j)}} indicates \id{P_1} produces version \id{x_j} \blue{concurrently} to version
      \id{x_i}.
    \end{itemize}
  \end{block}
\end{slide}
\begin{slide}{Monotonic reads}
  \begin{exampleblock}{Example}
    Automatically reading your personal calendar updates from different servers. Monotonic reads guarantees
    that the user sees all updates, no matter from which server the automatic reading takes place.
  \end{exampleblock}
  \begin{exampleblock}{Example}
    Reading (not modifying) incoming mail while you are on the move. Each time you connect to a different
    e-mail server, that server fetches (at least) all the updates from the server you previously visited.
  \end{exampleblock}
\end{slide}
\begin{slide}{Monotonic reads}
  \begin{block}{Definition}
    If a process reads the value of a data item \mathexpr{x}, any successive read operation
    on \mathexpr{x} by that process will always return that same or a more recent value.
  \end{block}
  \begin{centerfig}
    \includefigure[1]{07-18a} \\
    A monotonic-read consistent data store \\
  \end{centerfig}
  \vspace*{12pt}
  \begin{centerfig}
    \includefigure[1]{07-18b} \\
    A data store that does not provide monotonic reads
  \end{centerfig}
\end{slide}
\subsection{Monotonic writes}
\begin{slide}{Monotonic writes}
  \begin{exampleblock}{Example}
    Updating a program at server \id{S_2}, and ensuring that all components on which compilation and linking
    depends, are also placed at \id{S_2}.
  \end{exampleblock}
  \begin{exampleblock}{Example}
    Maintaining versions of replicated files in the correct order everywhere (propagate the previous version
    to the server where the newest version is installed).
  \end{exampleblock}
\end{slide}
\begin{slide}{Monotonic writes}
  \begin{block}{Definition}
    A write operation by a process on a data item \id{x} is completed before any successive write operation on
    \id{x} by the same process.
  \end{block}
  \begin{centerfig}
    \begin{tabular}{@{}c@{\hspace*{1cm}}c}
      \includefigure[1]{07-19a} & 
      \includefigure[1]{07-19b} \\
      OK & Not OK
    \end{tabular}
  \end{centerfig}
  \vspace*{12pt}
  \begin{centerfig}
    \begin{tabular}{@{}c@{\hspace*{1cm}}c}
      \includefigure[1]{07-19c} & 
      \includefigure[1]{07-19d} \\
      Not OK & OK
    \end{tabular}
  \end{centerfig}
\end{slide}
\subsection{Read your writes}
\begin{slide}{Read your writes}
  \begin{block}{Definition}
    The effect of a write operation by a process on a data item \id{x}, will always be seen by a successive read
    operation on \id{x} by the same process. 
  \end{block}
  \begin{centerfig}
    \includefigure[1]{07-20a} \\
    OK
  \end{centerfig}
  \vspace*{12pt}
  \begin{centerfig}
    \includefigure[1]{07-20b} \\
    Not OK
  \end{centerfig}
  \onslide
  \begin{exampleblock}{Example}
    Updating your Web page and guaranteeing that your Web browser shows the newest version instead of its
    cached copy.
  \end{exampleblock}
\end{slide}
\subsection{Writes follow reads}
\begin{slide}{Writes follow reads}
  \begin{block}{Definition}
    A write operation by a process on a data item \mathexpr{x} following a previous read operation on \mathexpr{x} by the same
    process, is guaranteed to take place on the same or a more recent value of \mathexpr{x} that was read.
  \end{block}
  \begin{centerfig}
    \includefigure[1]{07-21a} \\
    OK
  \end{centerfig}
  \vspace*{12pt}
  \begin{centerfig}
    \includefigure[1]{07-21b} \\
    Not OK
  \end{centerfig}
  \onslide
  \begin{exampleblock}{Example}
    See reactions to posted articles only if you have the original posting (a read ``pulls in'' the
    corresponding write operation).
  \end{exampleblock}
\end{slide}
\subsection{Example: client-centric consistency in ZooKeeper}
\begin{slide}{Example: ZooKeeper consistency}
  \begin{block}{Yet another model?}
    ZooKeeper's consistency model mixes elements of data-centric and client-centric models
  \end{block}
  \begin{exampleblock}{Take a naive example}
    \begin{centerfig}
      \includefigure{07-22}
    \end{centerfig}
  \end{exampleblock}
\end{slide}
\section{Replica management}
\subsection{Finding the best server location}
\begin{slide}{Replica placement}
  \begin{block}{Essence}
    Figure out what the best \mathexpr{K} places are out of \mathexpr{N} possible locations.
    \begin{itemize}
    \item Select best location out of \mathexpr{N-K} for which the \blue{average distance to clients is
      minimal}. Then choose the next best server. (\red{Note:} The first chosen location minimizes the
      average distance to all clients.) \blue{Computationally expensive}.
    \item Select the \mathexpr{K}-th largest \blue{autonomous system} and place a server at the
      best-connected host. \blue{Computationally expensive}.
    \item Position nodes in a \mathexpr{d}-dimensional geometric space, where distance reflects
      latency. Identify the \mathexpr{K} regions with highest density and place a server in every
      one. \blue{Computationally cheap}.
    \end{itemize}
  \end{block}
\end{slide}
\subsection{Content replication and placement}
\begin{slide}{Content replication}
  \begin{block}{Distinguish different processes}
    A process is capable of hosting a replica of an object or data:
    \begin{itemize}
    \item \blue{Permanent replicas:} Process/machine always having a replica
    \item \blue{Server-initiated replica:} Process that can dynamically host a replica on request of another
      server in the data store
    \item \blue{Client-initiated replica:} Process that can dynamically host a replica on request of a client
      (\red{client cache})
    \end{itemize}
  \end{block}
\end{slide}
\begin{slide}{Content replication}
  \begin{block}{The logical organization of different kinds of copies of a data store into three
    concentric rings}
    \begin{centerfig}
      \includefigure{07-24}
    \end{centerfig}
  \end{block}
\end{slide}
  \begin{slide}{Server-initiated replicas}
    \begin{exampleblock}{Counting access requests from different clients}
      \begin{centerfig}
        \includefigure{07-25}
      \end{centerfig}
      \begin{itemize}\tightlist
      \item Keep track of access counts per file, aggregated by considering server closest to requesting clients
      \item Number of accesses drops below threshold \mathexpr{D} \mathexpr{{\Rightarrow}} drop file
      \item Number of accesses exceeds threshold \mathexpr{R} \mathexpr{{\Rightarrow}} replicate file
      \item Number of access between \mathexpr{D} and \mathexpr{R} \mathexpr{{\Rightarrow}} migrate file
      \end{itemize}
    \end{exampleblock}
  \end{slide}
\subsection{Content distribution}
\begin{slide}{Content distribution}
  \begin{block}{Consider only a client-server combination}
    \begin{itemize}
    \item Propagate only \blue{notification/invalidation} of update (often used for caches)
    \item Transfer \blue{data} from one copy to another (distributed databases): \red{passive replication}
    \item Propagate the update \blue{operation} to other copies: \red{active replication}
    \end{itemize}
  \end{block}
  \begin{alertblock}{Note}
    No single approach is the best, but depends highly on available bandwidth and read-to-write ratio at
    replicas.
  \end{alertblock}
\end{slide}
\begin{slide}{Content distribution: client/server system}
  \begin{block}{A comparison between push-based and pull-based protocols in the case of
    multiple-client, single-server systems}
    \begin{itemize}
    \item \red{Pushing updates}: \blue{server-initiated approach}, in which update is propagated regardless
      whether target asked for it.
    \item \red{Pulling updates}: \blue{client-initiated approach}, in which client requests to be updated.
    \end{itemize}
    \begin{center}
      \footnotesize
      \renewcommand{\arraystretch}{1.2}
      \begin{tabular}{|l|l|l|}\hline
        \blue{Issue} & \blue{Push-based} & \blue{Pull-based} \\ \hline
        1: & List of client caches & None \\ \hline
        2: & Update (and possibly fetch update) & Poll and update \\ \hline
        3: & Immediate (or fetch-update time) & Fetch-update time  \\ \hline
        \multicolumn{3}{|l|}{\itshape 1: State at server} \\
        \multicolumn{3}{|l|}{\itshape 2: Messages to be exchanged} \\
        \multicolumn{3}{|l|}{\itshape 3: Response time at the client} \\ \hline
      \end{tabular}
    \end{center}
  \end{block}
\end{slide}
\begin{slide}{Content distribution}
  \vspace*{-6pt}
  \begin{block}{Observation} 
    We can dynamically switch between pulling and pushing using \blue{leases}: A contract in which the server
    promises to push updates to the client until the lease expires.
  \end{block}
  \begin{alertblock}{Make lease expiration time adaptive}
    \begin{itemize}
    \item\blue{Age-based leases}: An object that hasn't changed for a long time, will not change in the near
      future, so provide a long-lasting lease
    \item\blue{Renewal-frequency based leases}: The more often a client requests a specific object, the longer
      the expiration time for that client (for that object) will be
    \item\blue{State-based leases}: The more loaded a server is, the shorter the expiration times become
    \end{itemize}
  \end{alertblock}
  \onslide
  \begin{alertblock}{Question}
    Why are we doing all this?  
%%  Trying to reduce the server's state as much as possible while providing strong consistency.
  \end{alertblock}
\end{slide}
\subsection{Managing replicated objects}
\begin{slide}{Managing replicated objects}
  \begin{itemize}
  \item Prevent concurrent execution of multiple invocations on the same object: access to the
    internal data of an object has to be serialized. Using local locking mechanisms are sufficient.
  \item Ensure that all changes to the replicated state of the object are the same: no two independent
    method invocations take place on different replicas at the same time: we need \red{deterministic thread
      scheduling}. 
  \end{itemize}
  \begin{centerfig}
    \includefigure[0.71]{07-27}
  \end{centerfig}
\end{slide}
  \begin{slide}{Replicated-object invocations}
    \begin{block}{Problem when invocating a replicated object}
      \begin{centerfig}
        \includefigure{07-28}
      \end{centerfig}
    \end{block}
  \end{slide}
  \begin{slide}{Replicated-object invocations}
    \begin{centerfig}
      \begin{tabular}{@{}c@{\hspace*{1cm}}c}
        \includefigure{07-29a} &
        \includefigure{07-29b} \\
        Forwarding a request & Returning the reply \\
      \end{tabular}
    \end{centerfig}
  \end{slide}
\section{Consistency protocols}
\subsection{Sequential consistency: Primary-based protocols}
\begin{slide}{Primary-based protocols}
  \begin{block}{Primary-backup protocol}
    \begin{centerfig}
      \includefigure[0.65]{07-30}
    \end{centerfig}
  \end{block}
  \onslide
  \begin{exampleblock}{Example primary-backup protocol}
    Traditionally applied in distributed databases and file systems that require a high degree of fault
    tolerance. Replicas are often placed on the same LAN.
  \end{exampleblock}
\end{slide}
\begin{slide}{Primary-based protocols}
  \begin{block}{Primary-backup protocol with local writes}
    \begin{centerfig}
      \includefigure[0.65]{07-31}
    \end{centerfig}
  \end{block}
  \onslide
  \begin{exampleblock}{Example primary-backup protocol with local writes}
    Mobile computing in disconnected mode (ship all relevant files to user before disconnecting, and update
    later on).
  \end{exampleblock}
\end{slide}
\subsection{Sequential consistency: Replicated-write protocols}
\begin{slide}{Replicated-write protocols}
  \begin{block}{Quorum-based protocols} 
    Assume \mathexpr{N} replicas. Ensure that each operation is carried out in such a way that a majority vote
    is established: distinguish \blue{read quorum} \mathexpr{N_R} and \blue{write quorum}
    \mathexpr{N_W}. Ensure:
    \begin{enumerate}\firmlist
    \item \red{\mathexpr{N_R + N_W > N}} (prevent read-write conflicts)
    \item \red{\mathexpr{N_W > N / 2}} (prevent write-write conflicts)
    \end{enumerate}
  \end{block}

  \vspace*{12pt}
  \begin{center}
    \begin{tabular}{c@{\hspace*{1cm}}c@{\hspace*{1cm}}c}
      \includefigure{07-32a} &
      \includefigure{07-32b} &
      \includefigure{07-32c} \\
      Correct & Write-write conflict & Correct (ROWA)
    \end{tabular}
  \end{center}
\end{slide}
\subsection{Cache-coherence protocols}
\subsection{Implementing continuous consistency}
\begin{slide}{Continuous consistency: Numerical errors}
  \begin{block}{Principal operation}
    \begin{itemize}
    \item Every server \id{S_i} has a log, denoted as \red{\id{L_i}}.
    \item Consider a data item \id{x} and let \red{\mathexpr{val(\id{W})}} denote the \blue{numerical change} in its value
      after a write operation \id{W}. Assume that \[\forall \id{W}: val(\id{W}) > 0\]
    \item \id{W} is initially forwarded to one of the \mathexpr{N} replicas, denoted as \red{\mathexpr{origin(W)}}. \mathexpr{TW[i,j]} are
      the \blue{writes executed by server \mathexpr{S_i} that originated from \mathexpr{S_j}}:
      \[TW[i,j] = \sum \{ val(\id{W}) | origin(\id{W}) = \id{S_j} \mbox{\ \&\ } \id{W} \in \id{L_i} \}\]
    \end{itemize}
  \end{block}
\end{slide}
\begin{slide}{Continuous consistency: Numerical errors}
  \begin{alertblock}{Note}
    Actual value \mathexpr{v(t)} of \id{x}:
    \[v(t) = v_{init} + \sum_{k=1}^N TW[k,k]\]
    value \mathexpr{v_i} of \id{x} at server \id{S_i}:\vspace*{-0.25cm}
    \[v_i = v_{init} + \sum_{k=1}^N TW[i,k]\]
  \end{alertblock}
\end{slide}
\begin{slide}{Continuous consistency: Numerical errors}
  \begin{alertblock}{Problem}
    We need to ensure that \mathexpr{v(t) - v_i < \delta_i} for every server \id{S_i}.
  \end{alertblock}
  \onslide
  \begin{block}{Approach}
    Let every server \id{S_k} maintain a \blue{view} \mathexpr{TW_k[i,j]} of what it believes is the value of
    \mathexpr{TW[i,j]}. This information can be \blue{gossiped} when an update is propagated.
  \end{block}
  \onslide
  \begin{alertblock}{Note}
    \[0 \leq TW_k[i,j] \leq TW[i,j] \leq TW[j,j]\]
  \end{alertblock}
\end{slide}
\begin{slide}{Continuous consistency: Numerical errors}
  \begin{block}{Solution} 
    \id{S_k} sends operations from its log to \id{S_i} \blue{when it sees that \mathexpr{TW_k[i,k]} is getting too far from
    \mathexpr{TW[k,k]}}, in particular, when \[TW[k,k] - TW_k[i,k] > \delta_i/(N-1)\]
  \end{block}
  \onslide
  \begin{block}{Question}
    To what extent are we being \red{pessimistic} here: where does \mathexpr{\delta_i/(N-1)} come from?
  \end{block}
  \onslide
  \begin{alertblock}{Note}
    Staleness can be done analogously, by essentially keeping track of what has been seen last from \id{S_i} (see
    book).
  \end{alertblock}
\end{slide}
\subsection{Implementing client-centric consistency}
\begin{slide}{Implementing client-centric consistency}
  
  \begin{block}{Keeping it simple}
    Each write operation \id{W} is assigned a globally unique identifier by its \blue{origin server}. For each
    client, we keep track of two sets of writes:
    \begin{itemize}
    \item \red{Read set}: the (identifiers of the) writes relevant for that client's read operations
    \item \red{Write set}: the (identifiers of the) client's write operations.
    \end{itemize}
  \end{block}

  \onslide
  \begin{block}{Monotonic-read consistency}
    When client \id{C} wants to read at server \id{S}, \id{C} passes its read set. \id{S} can pull in any
    updates before executing the read operation, after which the read set is updated.
  \end{block}

  \onslide
  \begin{block}{Monotonic-write consistency}
    When client \id{C} wants to write at server \id{S}, \id{C} passes its write set. \id{S} can pull in any
    updates, executes them in the correct order, and then executes the write operation, after which the write
    set is updated.
  \end{block}

\end{slide}
\begin{slide}{Implementing client-centric consistency}
  
  \begin{block}{Read-your-writes consistency}
    When client \id{C} wants to read at server \id{S}, \id{C} passes its write set. \id{S} can pull in any
    updates before executing the read operation, after which the read set is updated.
  \end{block}

  \onslide
  \begin{block}{Writes-follows-reads consistency}
    When client \id{C} wants to write at server \id{S}, \id{C} passes its read set. \id{S} can pull in any
    updates, executes them in the correct order, and then executes the write operation, after which the write
    set is updated.
  \end{block}

\end{slide}
\section{Example: Caching and replication in the Web}
\begin{slide}{Example: replication in the Web}
  \begin{block}{Client-side caches}
    \begin{itemize}
    \item In the browser
    \item At a client's site, notably through a \blue{Web proxy}
    \end{itemize}
  \end{block}
  \begin{block}{Caches at ISPs}
    Internet Service Providers also place caches to (1) reduce cross-ISP traffic and (2) improve client-side
    performance. May get nasty when a request needs to pass many ISPs.
  \end{block}
\end{slide}
  \begin{slide}{Cooperative caching}
    \begin{centerfig}
      \includefigure[0.71]{07-35}
    \end{centerfig}
  \end{slide}
\begin{slide}{Web-cache consistency}
  \begin{block}{How to guarantee freshness?}
    To prevent that stale information is returned to a client:
    \begin{itemize}
    \item \red{Option 1}: let the cache contact the original server to see if content is still up to date.
    \item \red{Option 2}: Assign an expiration time \mathexpr{T_{expire}} that depends on how long ago the
      document was last modified when it is cached. If \mathexpr{T_{\mts{last\_modified}}} is the last
      modification time of a document (as recorded by its owner), and \mathexpr{T_{\mts{cached}}} is the time
      it was cached, then
      \[
      T_{\mts{expire}} = \alpha (T_{\mts{cached}} - T_{\mts{last\_modified}}) + T_{\mts{cached}}
      \]
      with \mathexpr{\alpha = 0.2}. Until \mathexpr{T_{\mts{expire}}}, the document is considered valid.
    \end{itemize}
  \end{block}
\end{slide}
\begin{slide}{Alternatives for caching and replication}
  \vspace*{-6pt}
  \begin{centerfig}
    \includefigure[0.83]{07-37}
  \end{centerfig}
  \begin{itemize}\tightlist
  \item \red{Database copy}: the edge has the same as the origin server
  \item \red{Content-aware cache}: check if a (normal query) can be answered with cached data. Requires that
    the server knows about which data is cached at the edge.
  \item \red{Content-blind cache}: store a query, and its result. When the exact same query is issued again,
    return the result from the cache.
  \end{itemize}
\end{slide}
\section{Summary}
\begin{slide}{Summary}
In this section on \emph{Consistency and Replication},
we discussed the following topics  
  \begin{itemize}\tightlist
  \item Data-centric consistency models
  \item Client-centric consistency models
  \item Replica Management
  \item Consistency Protocols
  \item Caching Example
  \end{itemize}
\end{slide}
