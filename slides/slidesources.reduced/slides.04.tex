\part{Communication}
\section{Foundations}
\subsection{Layered Protocols}
\begin{slide}{Basic networking model}
  \begin{block}{}
    \begin{center}
      \includefigure[0.65]{04-01}
    \end{center}
  \end{block}
  \begin{block}{Drawbacks}
    \begin{itemize}\tightlist
    \item Focus on message-passing only
    \item Violates access transparency - fails to hide the complexities of 
    accessing resources or services across different machines or locations, making it apparent to users or applications that 
    they are interacting with a distributed environment
    \end{itemize}
  \end{block}
\end{slide}
% \begin{slide}{Low-level layers}
%   \begin{block}{Recap}
%     \begin{itemize}\tightlist
%     \item \blue{Physical layer}: contains the specification and implementation of bits, and their transmission
%       between sender and receiver
%     \item \blue{Data link layer}: prescribes the transmission of a series of bits into a frame to allow for
%       error and flow control
%     \item \blue{Network layer}: describes how packets in a network of computers are to be \red{routed}.
%     \end{itemize}
%   \end{block}
%   \begin{block}{Observation} 
%     For many distributed systems, the lowest-level interface is that of the network layer. 
%   \end{block}
% \end{slide}
% \begin{slide}{Transport Layer}
%   \begin{alertblock}{Important} 
%     The transport layer provides the actual communication facilities for most distributed systems.
%   \end{alertblock}
%   \begin{block}{Standard Internet protocols}
%     \begin{itemize}\tightlist
%     \item TCP: connection-oriented, reliable, stream-oriented communication
%     \item UDP: unreliable (best-effort) datagram communication
%     \end{itemize}
%   \end{block}
% \end{slide}
\begin{slide}{Middleware ``layer"}
  \begin{block}{Observation} 
    Middleware is invented to provide \blue{common} services and protocols that can be used by many
    \blue{different} applications
    \begin{itemize}\tightlist
    \item A rich set of \blue{communication protocols}
    \item \blue{(Un)marshaling} of data, necessary for integrated systems
    \item \blue{Naming protocols}, to allow easy sharing of resources
    \item \blue{Security protocols} for secure communication
    \item \blue{Scaling mechanisms}, such as for replication and caching
    \end{itemize}
  \end{block}
  \begin{alertblock}{Note} 
    What remains are truly \blue{application-specific} protocols... \red{such as?}
  \end{alertblock}
\end{slide}
\begin{slide}{An adapted layering scheme}
  \begin{block}{}
    \begin{center}
      \includefigure{04-03}
    \end{center}
  \end{block}
\end{slide}
\subsection{Types of Communication}
\begin{slide}{Types of communication}
  \begin{block}{Distinguish...}
    \begin{center}
      \includefigure{04-04}
    \end{center}
    \begin{itemize}\tightlist
    \item \blue{Transient} versus \blue{persistent} communication
    \item \blue{Asynchronous} versus \blue{synchronous} communication
    \end{itemize}
  \end{block}
\end{slide}
\begin{slide}{Types of communication}
  \begin{block}{Transient versus persistent}
    \begin{center}
      \includefigure{04-04}
    \end{center}
    \begin{itemize}\tightlist
    \item \red{Transient communication:} Comm.\ server discards message when it cannot be delivered at the
      next server, or at the receiver.
    \item \red{Persistent communication:} A message is stored at a communication server as long as it takes to
      deliver it.
    \end{itemize}
  \end{block}
\end{slide}
\begin{slide}{Types of communication}
  \begin{block}{Places for synchronization}
    \begin{center}
      \includefigure{04-04}
    \end{center}
    \begin{itemize}\tightlist
    \item At \blue{request submission}
    \item At \blue{request delivery}
    \item After \blue{request processing}
    \end{itemize}
  \end{block}
\end{slide}
\begin{slide}{Client/Server}
  \begin{block}{Some observations}
    Client/Server computing is generally based on a model of \blue{transient synchronous communication}:
    \begin{itemize}\tightlist
    \item Client and server have to be active at the time of communication
    \item Client issues request and blocks until it receives reply
    \item Server essentially waits only for incoming requests, and subsequently processes them
    \end{itemize}
  \end{block}
  \onslide
  \begin{block}{Drawbacks synchronous communication}
    \begin{itemize}\tightlist
    \item Client cannot do any other work while waiting for reply
    \item Failures have to be handled immediately: the client is waiting
    \item The model may simply not be appropriate (mail, news)
    \end{itemize}
  \end{block}
\end{slide}
% \begin{slide}{Messaging}
%   \begin{block}{Message-oriented middleware}
%     Aims at high-level \blue{persistent asynchronous communication}:
%     \begin{itemize}\tightlist
%     \item Processes send each other messages, which are queued
%     \item Sender need not wait for immediate reply, but can do other things
%     \item Middleware often ensures  fault tolerance
%     \end{itemize}
%   \end{block}
% \end{slide}
\section{Remote procedure call}
\subsection{Basic RPC operation}
\begin{slide}{Basic RPC operation}
  \begin{block}{Observations}
    \begin{itemize}\tightlist
    \item Application developers are familiar with simple procedure model
    \item Well-engineered procedures operate in isolation (black box)
    \item There is no fundamental reason not to execute procedures on separate machine
    \end{itemize}
  \end{block}
  \begin{columns}
    \begin{column}{0.5\textwidth}
      \begin{alertblock}{Conclusion} 
        Communication between caller \& callee can be hidden by using procedure-call mechanism.
      \end{alertblock}
    \end{column}
    \begin{column}{0.5\textwidth}
      \includefigure{04-06}
    \end{column}
  \end{columns}
\end{slide}
\begin{slide}{Basic RPC operation}
  \begin{block}{}
    \begin{center}
      \includefigure{04-07}
    \end{center}
  \end{block}
    {\scriptsize
      \begin{tabular}{@{}ll}
        \begin{minipage}{0.45\textwidth}
          \begin{enumerate}\tightlist
          \item Client procedure calls client stub.
          \item Stub builds message; calls local OS.
          \item OS sends message to remote OS.
          \item Remote OS gives message to stub.
          \item Stub unpacks parameters; calls server.
          \end{enumerate}
        \end{minipage} &
        \begin{minipage}{0.55\textwidth}
          \begin{enumerate}\tightlist\setcounter{enumi}{5}
          \item Server does local call; returns result to stub.
          \item Stub builds message; calls  OS.
          \item OS sends message to client's OS.
          \item Client's OS gives message to stub.
          \item Client stub unpacks result; returns to client.
          \end{enumerate}
        \end{minipage}
      \end{tabular}
    }
\end{slide}
\subsection{Parameter passing}
\begin{slide}{RPC: Parameter passing}
  \begin{block}{There's more than just wrapping parameters into a message}
    \begin{itemize}\tightlist
    \item Client and server machines may have \blue{different data representations} (think of byte ordering)
    \item Wrapping a parameter means \blue{transforming a value into a sequence of bytes}
    \item Client and server have to \blue{agree on the same encoding}:
    \end{itemize}
    \begin{itemize}\tightlist
    \item How are \blue{basic data values} represented (integers, floats, characters)
    \item How are \blue{complex data values} represented (arrays, unions)
    \end{itemize}
  \end{block}
  \begin{alertblock}{Conclusion}
    Client and server need to \blue{properly interpret messages}, transforming them into machine-dependent
    representations. 
  \end{alertblock}
\end{slide}
\begin{slide}{RPC: Parameter passing}
  \begin{block}{Some assumptions}
      \begin{itemize}\tightlist
      \item \blue{Copy in/copy out} semantics: while procedure is executed, nothing can be assumed about parameter
        values.
      \item \blue{All} data that is to be operated on is passed by parameters. Excludes passing
        \blue{references to (global) data}.
      \end{itemize}
  \end{block}
  \onslide
  \begin{alertblock}{Conclusion} 
    Full access transparency cannot be realized.
  \end{alertblock}
  \onslide
  \begin{block}{A remote reference mechanism enhances access transparency}
    \begin{itemize}\tightlist
    \item Remote reference offers \blue{unified access} to remote data
    \item Remote references can be \blue{passed as parameter} in RPCs
    \item \red{Note}: stubs can sometimes be used as such references
    \end{itemize}
  \end{block}
\end{slide}
\subsection{RPC-based application support}
\subsection{Variations on RPC}
\begin{slide}{Asynchronous RPCs}
  \begin{block}{Essence} 
    Try to get rid of the strict request-reply behavior, but let the client continue without waiting for an
    answer from the server.
  \end{block}
  \begin{block}{}
    \begin{center}
      \includefigure{04-15}
    \end{center}
  \end{block}
\end{slide}
\begin{slide}{Sending out multiple RPCs}
  \begin{block}{Essence}
    Sending an RPC request to a group of servers.
  \end{block}
  \begin{block}{}
    \begin{center}
      \includefigure{04-16}
    \end{center}
  \end{block}
\end{slide}
\section{Message-oriented communication}
\subsection{Simple transient messaging with sockets}
\begin{slide}{Transient messaging: sockets}
  \begin{block}{Berkeley socket interface}\scriptsize
    \begin{center}
      \renewcommand{\arraystretch}{1}
      \begin{tabular}{|l|>{\RRCOL}p{0.7\textwidth}|} \hline
        \textbf{Operation} & \textbf{Description} \\ \whline
        \code{socket}  & Create a new communication end point   \\
        \code{bind}    & Attach a local address to a socket   \\
        \code{listen}  & Tell operating system what the maximum number of pending connection requests should be \\
        \code{accept}  & Block caller until a connection request arrives   \\
        \code{connect} & Actively attempt to establish a connection   \\
        \code{send}    & Send some data over the connection   \\
        \code{receive} & Receive some data over the connection   \\
        \code{close}   & Release the connection  \\ \hline
      \end{tabular}
    \end{center}
  \end{block}
  \begin{block}{}
    \begin{center}
      \includefigure{04-18}
    \end{center}
  \end{block}
\end{slide}
  % \begin{slide}{Sockets: Python code}
  %   \vspace*{-6pt}
  %   \begin{block}{Server}
  %     \begin{tabular}{@{\hspace*{1em}}l}
  %       \includelisting{04-19/server-book}
  %     \end{tabular}
  %   \end{block}
  %   \begin{block}{Client}
  %     \begin{tabular}{@{\hspace*{1em}}l}
  %       \includelisting{04-19/client-book}
  %     \end{tabular}
  %   \end{block}
  % \end{slide}
\subsection{Advanced transient messaging}
\begin{slide}{Making sockets easier to work with}
  \begin{block}{Observation}
    Sockets are rather low level and programming mistakes are easily made. However, the way that they are used
    is often the same (such as in a client-server setting).
  \end{block}
  \begin{block}{Alternative: ZeroMQ}
    Provides a higher level of expression by \red{pairing} sockets: one for sending messages at process \id{P}
    and a corresponding one at process \id{Q} for receiving messages. All communication is \red{asynchronous}.
  \end{block}
  \begin{block}{Three patterns}
    \begin{itemize}\tightlist
    \item Request-reply
    \item Publish-subscribe
    \item Pipeline
    \end{itemize}
  \end{block}
\end{slide}
  \begin{slide}{Request-reply}
    \begin{centerfig}
      \includelisting{04-21/run-book}
    \end{centerfig}
  \end{slide}
  \begin{slide}{Publish-subscribe}
    \begin{centerfig}
      \includelisting{04-22/run-book}
    \end{centerfig}
  \end{slide}
  \begin{slide}{Pipeline}
    \begin{centerfig}
      \includelisting{04-23/run-book}
    \end{centerfig}
  \end{slide}
\begin{slide}{MPI: When lots of flexibility is needed}
  \begin{block}{Representative operations}
    \begin{center}
      \begin{tabular}{|l|>{\RRCOL}p{0.7\textwidth}|} \hline
        \textbf{Operation} & \textbf{Description} \\ \whline
        \code{MPI\_BSEND}    & Append outgoing message to a local send buffer \\
        \code{MPI\_SEND}     & Send a message and wait until copied to local or remote buffer \\
        \code{MPI\_SSEND}    & Send a message and wait until transmission starts \\
        \code{MPI\_SENDRECV} & Send a message and wait for reply \\
        \code{MPI\_ISEND}    & Pass reference to outgoing message, and continue \\
        \code{MPI\_ISSEND}   & Pass reference to outgoing message, and wait until receipt starts \\
        \code{MPI\_RECV}     & Receive a message; block if there is none \\
        \code{MPI\_IRECV}    & Check if there is an incoming message, but do not block \\ \hline
      \end{tabular}
    \end{center}
  \end{block}
\end{slide}
\subsection{Message-oriented persistent communication}
\begin{slide}{Queue-based messaging}
  \begin{block}{Four possible combinations}
    \begin{center}
      \newcommand{\hs}{\hspace*{1.5cm}}
      \begin{tabular}{c@{\hs}c@{\hs}c@{\hs}c}
        \includefigure{04-25a} &
        \includefigure{04-25b} &
        \includefigure{04-25c} &
        \includefigure{04-25d} 
      \end{tabular}
    \end{center}
  \end{block}
\end{slide}
\begin{slide}{Message-oriented middleware}
  \begin{block}{Essence} 
    Asynchronous persistent communication through support of middleware-level \blue{queues}. Queues
    correspond to buffers at communication servers.
  \end{block}
  \begin{block}{Operations}
    \begin{center}
      \begin{tabular}{|l|>{\RRCOL}p{0.7\textwidth}|} \hline
        \textbf{Operation} & \textbf{Description} \\ \whline
        \code{PUT}    & Append a message to a specified queue \\
        \code{GET}    & Block until the specified queue is nonempty, and remove the first message \\
        \code{POLL}   & Check a specified queue for messages, and remove the first. Never block \\
        \code{NOTIFY} & Install a handler to be called when a message is put into the specified queue \\ \hline
      \end{tabular}
    \end{center}
  \end{block}
\end{slide}
\begin{slide}{General model}
  \begin{block}{Queue managers}
    Queues are managed by \red{queue managers}. An application can put messages only into a \blue{local}
    queue. Getting a message is possible by extracting it from a \blue{local} queue only
    \mathexpr{\Rightarrow} queue managers need to \blue{route} messages.
  \end{block}
  \begin{block}{Routing}
    \begin{center}
      \includefigure{04-27}
    \end{center}
  \end{block}
\end{slide}
\begin{slide}{Message broker}
  \begin{block}{Observation} 
    Message queuing systems assume a \blue{common messaging protocol}: all applications agree on message
    format (i.e., structure and data representation)
  \end{block}
  \begin{block}{Broker handles application heterogeneity in an MQ system}
    \begin{itemize}\tightlist
    \item Transforms incoming messages to target format
    \item Very often acts as an \blue{application gateway}
    \item May provide \blue{subject-based} routing capabilities (i.e., \blue{publish-subscribe} capabilities)
    \end{itemize}
  \end{block}
\end{slide}
\begin{slide}{Message broker: general architecture}
  \begin{centerfig}
    \includefigure{04-28}
  \end{centerfig}
\end{slide}
\subsection{Example: Advanced Message Queuing Protocol (AMQP)}
\begin{slide}{Example: AMQP}
  \begin{block}{Lack of standardization}
    \blue{Advanced Message-Queuing Protocol} was intended to play the same role as, for example, TCP in
    networks: a protocol for high-level messaging with different implementations.
  \end{block}
  \begin{centerfig}
    \includefigure{04-29}
  \end{centerfig}
  \begin{block}{Basic model}
    Client sets up a (stable) \blue{connection}, which is a container for serveral (possibly ephemeral)
    \blue{one-way channels}. Two one-way channels can form a \blue{session}. A \blue{link} is akin to a
    socket, and maintains state about message transfers.
  \end{block}
\end{slide}
\begin{slide}{Example: AMQP-based producer}
  \begin{centerfig}
    \includelisting{04-30/producer-book}
  \end{centerfig}
\end{slide}
\begin{slide}{Example: AMQP-based consumer}
  \begin{centerfig}
    \includelisting{04-30/consumer-book}
  \end{centerfig}
\end{slide}
\section{Multicast communication}
\subsection{Application-level tree-based multicasting}
\begin{slide}{Application-level multicasting}
  \begin{block}{Essence} 
    Organize nodes of a distributed system into an \blue{overlay network} and use that network to disseminate
    data:
    \begin{itemize}\tightlist
    \item Oftentimes a \red{tree}, leading to unique paths
    \item Alternatively, also \red{mesh networks}, requiring a form of \blue{routing}
    \end{itemize}
  \end{block}
\end{slide}
  \begin{slide}{Application-level multicasting in Chord}
    \begin{block}{Basic approach}
      \begin{enumerate}\tightlist
      \item Initiator generates a \blue{multicast identifier} \id{mid}.
      \item Lookup \blue{\id{succ(mid)}}, the node responsible for \id{mid}.
      \item Request is routed to \id{succ(mid)}, which will become the \blue{root}.
      \item If \id{P} wants to join, it sends a \blue{join} request to the root.
      \item When request arrives at \id{Q}:
      
        \begin{itemize}\tightlist
	    \item \id{Q} has not seen a join request before \mathexpr{\Rightarrow} it becomes \blue{forwarder};
          \id{P} becomes child of \id{Q}. \red{Join request continues to be forwarded}.
	    \item \id{Q} knows about tree \mathexpr{\Rightarrow} \id{P} becomes child of \id{Q}. \red{No need to
          forward join request anymore}.
        \end{itemize}
      \end{enumerate}
    \end{block}
  \end{slide}
\begin{slide}{ALM: Some costs}
  \begin{block}{Different metrics}
    \begin{center}
      \includefigure{04-31}
    \end{center}
    \begin{itemize}\tightlist
    \item \red{Link stress}: How often does an ALM message cross the same physical link? \blue{Example:}
      message from \id{A} to \id{D} needs to cross \mathexpr{\langle \id{Ra},\id{Rb} \rangle} twice.
    \item \red{Stretch}: Ratio in delay between ALM-level path and network-level path. \blue{Example:}
      messages \id{B} to \id{C} follow path of length 73 at ALM, but 47 at network level
      \mathexpr{\Rightarrow} stretch = 73/47.
    \end{itemize}
  \end{block}
\end{slide}
\subsection{Flooding-based multicasting}
\begin{slide}{Flooding}
  \begin{block}{Essence}
    \id{P} simply sends a message \id{m} to each of its neighbors. Each neighbor will forward that
    message, except to \id{P}, and only if it had not seen \id{m} before.
  \end{block}
  \begin{centerfig}
    \includefigure[0.71]{04-32}
  \end{centerfig}
  \onslide
  \begin{block}{Variation}
    Let \id{Q} forward a message with a certain probability \mathexpr{p_{flood}}, possibly even dependent
    on its own number of neighbors (i.e., \red{node degree}) or the degree of its neighbors.
  \end{block}
\end{slide}
\subsection{Gossip-based data dissemination}
\begin{slide}{Epidemic protocols}
  \begin{block}{Assume there are no write--write conflicts}
    \begin{itemize}\tightlist
    \item Update operations are performed at a single server
    \item A replica passes updated state to only a few neighbors
    \item Update propagation is lazy, i.e., not immediate
    \item Eventually, each update should reach every replica
    \end{itemize}
  \end{block}
  \begin{block}{Two forms of epidemics}
    \begin{itemize}\tightlist
    \item \red{Anti-entropy}: Each replica regularly chooses another replica at random, and exchanges state
      differences, leading to identical states at both afterwards
    \item \red{Rumor spreading}: A replica which has just been updated (i.e., has been \blue{contaminated}),
      tells several other replicas about its update (contaminating them as well).
    \end{itemize}
  \end{block}
\end{slide}
\begin{slide}{Anti-entropy}
  \begin{block}{Principle operations}
    \begin{itemize}\tightlist
    \item A node \id{P} selects another node \id{Q} from the system at random.
    \item \red{Pull}: \id{P} only pulls in new updates from \id{Q}
    \item \red{Push}: \id{P} only pushes its own updates to \id{Q}
    \item \red{Push-pull}: \id{P} and \id{Q} send updates to each other
    \end{itemize}
  \end{block}
  \begin{block}{Observation} 
    For push-pull it takes \mathexpr{{\cal O} (log(N))} rounds to disseminate updates to all \mathexpr{N}
    nodes (\red{round} = when every node has taken the initiative to start an exchange).
  \end{block}
\end{slide}
  \begin{slide}{Anti-entropy: analysis}
    \begin{block}{Basics}
      Consider a single source, propagating its update. Let \mathexpr{p_i} be the probability that a node has
      not received the update after the \mathexpr{i^{\mbox{\scriptsize\itshape th}}} round.
    \end{block}
    \begin{block}{Analysis: staying ignorant}
      \begin{itemize}\firmlist
      \item With \blue{pull}, \mathexpr{p_{i+1} = (p_i)^2}: the node was not updated during the
        \mathexpr{i^{\mbox{\scriptsize\itshape th}}} round and should contact another ignorant node during the
        next round.
      \item With \blue{push}, \mathexpr{p_{i+1} = p_i(1 - \frac{1}{N-1})^{(N-1)(1-p_i)} \approx p_i e^{-1}} (for
        small \mathexpr{p_i} and large \mathexpr{N}): the node was ignorant during the
        \mathexpr{i^{\mbox{\scriptsize\itshape th}}} round and no updated node chooses to contact it during
        the next round.
      \item With \blue{push-pull}: \mathexpr{(p_i)^2 \cdot (p_i e^{-1})}
      \end{itemize}
    \end{block}
  \end{slide}
  \begin{slide}{Anti-entropy performance}
    \begin{centerfig}
      \begin{tabular}{c}
        \includefigure[0.71]{04-35a} \\
        \includefigure[0.71]{04-35b} 
      \end{tabular}
    \end{centerfig}
  \end{slide}
\begin{slide}{Rumor spreading}
  \newcommand{\pstop}{p_{\mbox{\scriptsize\itshape stop}}}
  \begin{block}{Basic model} 
    A server \id{S} having an update to report, contacts other servers. If a server is contacted to which the
    update has already propagated, \id{S} stops contacting other servers with probability \mathexpr{\pstop}.
  \end{block}
  \begin{block}{Observation}
    If \mathexpr{s} is the fraction of ignorant servers (i.e., which are unaware of the update), it can be
    shown that with many servers
    \[ s = e^{-(1/{\pstop} + 1)(1 - s)}\]
  \end{block}
\end{slide}
  \begin{slide}{Formal analysis}
    \newcommand{\pstop}{p_{\mbox{\scriptsize\itshape stop}}}
    \begin{block}{Notations}
      Let \red{\mathexpr{s}} denote fraction of nodes that have not yet been updated (i.e.,
      \blue{susceptible}; \red{\mathexpr{i}} the fraction of updated (\blue{infected}) and active nodes; and
      \red{\mathexpr{r}} the fraction of updated nodes that gave up (\blue{removed}).
    \end{block}
    \begin{block}{From theory of epidemics}
      \[\begin{array}{llcl}
      (1) & {ds}/{dt} & = & -s \cdot i \\
      (2) & {di}/{dt} & = & s \cdot i - \pstop \cdot (1-s) \cdot  i \\ 
      \Rightarrow & di/ds & = & -(1 + \pstop) + \frac{\pstop}{s} \\
      \Rightarrow & i(s) & = & -(1 + \pstop) \cdot s  + \pstop \cdot \ln(s) + C
      \end{array}
      \]
    \end{block}
    \begin{block}{Wrap up}
      \mathexpr{i(1) = 0 \Rightarrow C = 1 + \pstop \Rightarrow i(s) = (1+\pstop) \cdot (1-s) + \pstop \cdot
        \ln(s)}. We are looking for the case \mathexpr{i(s) = 0}, which leads to
      \mathexpr{s = e^{-(1/\pstop + 1)(1-s)}}
    \end{block}
  \end{slide}
\begin{slide}{Rumor spreading}
  \begin{block}{The effect of stopping}
    \begin{columns}
      \begin{column}{0.4\textwidth}
        \includefigure[0.71]{04-36}
      \end{column}
      \begin{column}{0.4\textwidth}
        \newcommand{\pstop}{p_{\mbox{\scriptsize\itshape stop}}}
        \footnotesize
        \[
        \begin{array}{|lcr|} \hline
	      \multicolumn{3}{|c|}{\mbox{Consider 10,000 nodes}} \\ \hline
	      1/\pstop   &   s         & N_s \\ \hline
	      1          &   0.203188  & 2032 \\
	      2          &   0.059520  & 595 \\
	      3          &   0.019827  & 198 \\
	      4          &   0.006977  & 70 \\
	      5          &   0.002516  & 25 \\ 
	      6          &   0.000918  & 9 \\
	      7          &   0.000336  & 3 \\ \hline
	    \end{array}
        \]
      \end{column}
    \end{columns}
  \end{block}
  \onslide
  \begin{alertblock}{Note}
    If we really have to ensure that all servers are eventually updated, rumor spreading alone is not enough
  \end{alertblock}
\end{slide}
\begin{slide}{Deleting values}
  \begin{alertblock}{Fundamental problem} 
    We cannot remove an old value from a server and expect the removal to propagate. Instead, mere removal
    will be undone in due time using epidemic algorithms
  \end{alertblock}
  \begin{block}{Solution} 
    Removal has to be registered as a special update by inserting a \blue{death certificate}
  \end{block}
\end{slide}
\begin{slide}{Deleting values}
  \begin{block}{When to remove a death certificate (it is not allowed to stay for ever)}
    \begin{itemize}\tightlist
    \item Run a global algorithm to detect whether the removal is known everywhere, and then collect the death
      certificates (looks like garbage collection)
    \item Assume death certificates propagate in finite time, and associate a maximum lifetime for a
      certificate (can be done at risk of not reaching all servers)
    \end{itemize}
  \end{block}
  \begin{alertblock}{Note} 
    It is necessary that a removal actually reaches all servers.
  \end{alertblock}
\end{slide}
\section{Summary}
\begin{slide}{Summary and Conclusions}
We have discussed some important communication principles
in Distributed Systems, namely:
\begin{itemize}
  \item Types of Communication
  \item Basic RPC
  \item Asynchronous messaging patterns
  \item Message-oriented middleware (MPI)
  \item Application-level multicasting
\end{itemize}  
\end{slide}