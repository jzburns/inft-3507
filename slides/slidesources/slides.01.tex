\part{Introduction}
\section{From networked systems to distributed systems}
\subsection{Distributed versus decentralized systems}
\begin{slide}{Distributed versus Decentralized}
  \begin{block}{What many people state}
  \end{block}
  \onslide<2->
  \begin{alertblock}{When does a decentralized system become distributed?}
    \begin{itemize}\tightlist
    \item Adding 1 link between two nodes in a decentralized system?
    \item Adding 2 links between two other nodes?
    \item In general: adding \mathexpr{k>0} links....? 
    \end{itemize}
  \end{alertblock}
\end{slide}
\begin{slide}{Alternative approach}
  \begin{block}{Two views on realizing distributed systems}
    \begin{itemize}
    \item \blue{Integrative view}: connecting existing networked computer systems into a larger a system.
    \item \blue{Expansive view}: an existing networked computer systems is extended with additional
      computers 
    \end{itemize}
  \end{block}
  \begin{block}{Definitions}
    \begin{itemize}
    \item A \blue{decentralized system} is a networked computer system in which processes and resources are
      \red{necessarily} spread across multiple computers.
    \item A \blue{distributed system} is a networked computer system in which processes and resources are
      \red{sufficiently} spread across multiple computers.
    \item A \blue{distributed system} is a networked computer system in which processes and resources are
      \red{sufficiently} spread across multiple computers.
    \end{itemize}
    Decentralized computing involves independent nodes 
    operating without a central authority, each making 
    \emph{autonomous} decisions    
  \end{block}
\end{slide}
  \begin{slide}{Some common misconceptions}
    \begin{block}{Centralized solutions do not scale}
      Make distinction between \red{logically} and \red{physically} centralized. The root of the Domain Name
      System:\vspace*{-6pt}
      \begin{itemize}\tightlist
      \item logically centralized
      \item physically (massively) distributed
      \item decentralized across several organizations
      \end{itemize}
    \end{block}
    \onslide<2->
    \begin{block}{Centralized solutions have a single point of failure}
      Generally not true (e.g., the root of DNS). A single point of
      failure is often:\vspace*{-6pt}
      \begin{itemize}\tightlist
      \item easier to manage
      \item easier to make more robust
      \end{itemize}
    \end{block}
    \onslide<3->
    \begin{alertblock}{Important}
      There are many, poorly founded, misconceptions regarding \red{scalability}, \red{fault tolerance},
      \red{security}, etc. We need to develop skills by which distributed systems can be readily understood so
      as to judge such misconceptions.
    \end{alertblock}
  \end{slide}
\subsection{Why making the distinction is relevant}
\subsection{Studying distributed systems}
\begin{slide}{Perspectives on distributed systems}
  \begin{block}{Distributed systems are complex: take persepctives}
    \begin{itemize}\firmlist
    \item \blue{Architecture}: common organizations
    \item \blue{Process}: what kind of processes, and their relationships
    \item \blue{Communication}: facilities for exchanging data
    \item \blue{Coordination}: application-independent algorithms
    \item \blue{Naming}: how do you identify resources?
    \item \blue{Consistency} and \blue{replication}: performance requires of data, which need to be \red{the
      same}
    \item \blue{Fault tolerance}: keep running in the presence of partial failures
    \item \blue{Security}: ensure authorized access to resources
    \end{itemize}
  \end{block}
\end{slide}
\section{Design goals}
\begin{slide}{What do we want to achieve?}
  \begin{block}{Overall design goals}
    \begin{itemize}
    \item Support sharing of resources
    \item Distribution transparency
    \item Openness
    \item Scalability
    \end{itemize}
  \end{block}
\end{slide}
\subsection{Resource sharing}
\begin{slide}{Sharing resources}
  \begin{exampleblock}{Canonical examples}
    \begin{itemize}\firmlist
    \item Cloud-based shared storage and files
    \item Peer-to-peer assisted multimedia streaming
    \item Shared mail services (think of outsourced mail systems)
    \item Shared Web hosting (think of content distribution networks)
    \end{itemize}
  \end{exampleblock}
  \begin{block}{Observation}
    \begin{quote}\itshape
      ``The network is the computer''
    \end{quote}
    (quote from John Gage, then at Sun Microsystems)
  \end{block}
\end{slide}
\subsection{Distribution transparency}
\begin{slide}{Distribution transparency}
  \begin{centerfig}
    \includefigure{01-02}
  \end{centerfig}

  \begin{block}{What is transparency?}
    \itshape The phenomenon by which a distributed system attempts to \blue{hide} the fact that its processes
    and resources are \blue{physically distributed across multiple computers}, possibly \blue{separated by
      large distances}.
  \end{block}
  \onslide<2->
  \begin{block}{Observation}
    Distribution transparancy is handled through many different techniques in a layer between applications and
    operating systems: a \red{middleware layer}
  \end{block}

\end{slide}
\begin{slide}{Distribution transparency}
  \begin{block}{Types}
    \begin{center}
      \sffamily\small \renewcommand{\arraystretch}{1.1}
      \begin{tabular}{|l|>{\RRCOL}p{0.7\textwidth}|} \hline
        \blue{Transparency} & \blue{Description} \\ \whline
        \red{Access}        & Hide differences in data representation and how an object is accessed \\ \hline
        \red{Location}      & Hide where an object is located \\ \hline
        \red{Relocation}    & Hide that an object may be moved to another location while in use \\ \hline
        \red{Migration}     & Hide that an object may move to another location \\ \hline
        \red{Replication}   & Hide that an object is replicated \\ \hline
        \red{Concurrency}   & Hide that an object may be shared by several independent users \\ \hline
        \red{Failure}       & Hide the failure and recovery of an object \\ \hline
      \end{tabular}
    \end{center}
  \end{block}
\end{slide}
\begin{slide}{Degree of transparency}
  \begin{block}{Aiming at full distribution transparency may be too much}
    \begin{itemize}\firmlist
    \item<2-> There are communication latencies that cannot be hidden
    \item<3-> \red{Completely hiding failures} of networks and nodes is (theoretically and practically)
      \blue{impossible}
      \begin{itemize}\tightlist
      \item You cannot distinguish a slow computer from a failing one
      \item You can never be sure that a server actually performed an operation before a crash
      \end{itemize}
    \item<4-> Full transparency will \red{cost performance}, exposing distribution of the system
      \begin{itemize}\tightlist
      \item Keeping replicas \blue{exactly} up-to-date with the master \blue{takes time}
      \item Immediately flushing write operations to disk for fault tolerance
      \end{itemize}
    \end{itemize}
  \end{block}
\end{slide}
\begin{slide}{Degree of transparency}
\begin{exampleblock}{Exposing distribution may be good}
    \begin{itemize}
    \item Making use of location-based services (finding your nearby friends)
    \item When dealing with users in different time zones
    \item When it makes it easier for a user to understand what's going on (when e.g., a server does not
      respond for a long time, report it as failing).
    \end{itemize}
  \end{exampleblock}
  \onslide<2->
  \begin{block}{Conclusion}
    Distribution transparency is a nice goal, but achieving it is a different story, and it should often not
    even be aimed at.
  \end{block}
\end{slide}
\subsection{Openness}
\begin{slide}{Openness of distributed systems}
  \begin{block}{Open distributed system}
    \itshape A system that \blue{offers components} that can easily be used by, or \blue{integrated into
      other systems}. An open distributed system itself will often consist of components that originate from
    elsewhere.
  \end{block}

  \begin{block}{What are we talking about?}
    Be able to interact with services from other open systems, irrespective of the underlying environment:
    \begin{itemize}\tightlist
    \item Systems should conform to well-defined \blue{interfaces}
    \item Systems should easily \blue{interoperate}
    \item Systems should support \blue{portability} of applications
    \item Systems should be easily \blue{extensible}
    \end{itemize}
  \end{block}
\end{slide}
\begin{slide}{Policies versus mechanisms}
  \begin{block}{Implementing openness: policies}
    \begin{itemize}\firmlist
    \item What level of consistency do we require for client-cached data?
    \item Which operations do we allow downloaded code to perform?
    \item Which QoS requirements do we adjust in the face of varying bandwidth?
    \item What level of secrecy do we require for communication?
    \end{itemize}
  \end{block}
  \begin{block}{Implementing openness: mechanisms}
    \begin{itemize}\firmlist
    \item Allow (dynamic) setting of caching policies
    \item Support different levels of trust for mobile code
    \item Provide adjustable QoS parameters per data stream
    \item Offer different encryption algorithms
    \end{itemize}
  \end{block}
\end{slide}
  \begin{slide}{On strict separation}
    \begin{block}{Observation}
      The stricter the separation between policy and mechanism, the more we need to ensure proper
      mechanisms, potentially leading to many configuration parameters and complex management.
    \end{block}

    \begin{alertblock}{Finding a balance}
      Hard-coding policies often simplifies management, \red{and} reduces complexity at the price of less
      flexibility. There is no obvious solution.
    \end{alertblock}
  \end{slide}
\subsection{Dependability}
\begin{slide}{Dependability}
  \begin{block}{Basics}
    A \blue{component} provides \blue{services} to \blue{clients}.  To provide services, the component may
    require the services from other components \mathexpr{\Rightarrow} a component may \red{depend} on some
    other component.
  \end{block}
  \begin{block}{Specifically}
    A component \id{C} depends on \id{C^*} if the \blue{correctness} of \id{C}'s behavior depends on the
    correctness of \id{C^*}'s behavior. (Components are processes or channels.)
  \end{block}
\end{slide}
\begin{slide}{Dependability}
  \begin{block}{Requirements related to dependability}
    \begin{center}
      \begin{tabular}{|l|l|}\hline
        \textbf{Requirement} & \textbf{Description} \\ \whline
        \red{Availability}	 & Readiness for usage 							\\ \hline
        \red{Reliability}    & Continuity of service delivery				\\ \hline
        \red{Safety}	     & Very low probability of catastrophes			\\ \hline
        \red{Maintainability} & How easy can a failed system be repaired    \\ \hline
      \end{tabular}
    \end{center}
  \end{block}
\end{slide}
\begin{slide}{Reliability versus availability}
  \begin{block}{Reliability \mathexpr{R(t)} of component \id{C}}
    Conditional probability that \id{C} has been functioning correctly during \mathexpr{[0,t)} given
      \id{C} was functioning correctly at the time \mathexpr{T=0}.
  \end{block}

  \begin{block}{Traditional metrics}
    \begin{itemize}\firmlist
    \item \red{Mean Time To Failure} (\blue{\MTTF}): The average time until a component fails.
    \item \red{Mean Time To Repair} (\blue{\MTTR}): The average time needed to repair a component.
    \item \red{Mean Time Between Failures} (\blue{\MTBF}): Simply \MTTF\ + \MTTR.
    \end{itemize}
  \end{block}
\end{slide}
\begin{slide}{Terminology}
  \begin{block}{Failure, error, fault}
    \begin{center}
      \begin{tabular}{|l|>{\RRCOL}p{0.4\textwidth}|>{\RRCOL}p{0.3\textwidth}|}\hline
        \textbf{Term} & \textbf{Description} & \textbf{Example} \\ \whline
        \red{Failure} & A component is not living up to its specifications & Crashed program \\ \hline
        \red{Error}   & Part of a component that can lead to a failure     & Programming bug \\ \hline
        \red{Fault}   & Cause of an error                                  & Sloppy programmer \\ \hline
      \end{tabular}
    \end{center}
  \end{block}
\end{slide}
\begin{slide}{Terminology}
  \begin{block}{Handling faults}
    \begin{center}
      \begin{tabular}{|>{\RRCOL}p{0.2\textwidth}|>{\RRCOL}p{0.3\textwidth}|>{\RRCOL}p{0.3\textwidth}|}\hline
        \textbf{Term} & \textbf{Description} & \textbf{Example} \\ \whline
        \red{Fault prevention} 
        & Prevent the occurrence of a fault 
        & Don't hire sloppy programmers \\ \hline
        \red{Fault tolerance}  
        & Build a component and make it mask the occurrence of a fault 
        & Build each component by two independent programmers \\ \hline
        \red{Fault removal}
        & Reduce the presence, number, or seriousness of a fault
        & Get rid of sloppy programmers \\ \hline
        \red{Fault forecasting} 
        & Estimate current presence, future incidence, and consequences of faults
        & Estimate how a recruiter is doing when it comes to hiring sloppy programmers \\ \hline
      \end{tabular}
    \end{center}
  \end{block}
\end{slide}
\subsection{Security}
\begin{slide}{On security}
  \begin{alertblock}{Observation}
    A distributed system that is not secure, is not dependable
  \end{alertblock}
  \onslide<2->
  \begin{block}{What we need}
    \begin{itemize}\firmlist
    \item \red{Confidentiality}: information is disclosed only to authorized parties
    \item \red{Integrity}: Ensure that alterations to assets of a system can be made only in an authorized way
    \end{itemize}
  \end{block}
  \onslide<3->
  \begin{block}{Authorization, Authentication, Trust}
    \begin{itemize}\firmlist
    \item \blue{Authentication}: verifying the correctness of a claimed identity
    \item \blue{Authorization}: does an identified entity has proper access rights?
      \item \blue{Trust}: one entity can be assured that another will perform particular actions
        according to a specific expectation
    \end{itemize}
  \end{block}
\end{slide}
\begin{slide}{Security mechanisms}
  \begin{block}{Keeping it simple}
    It's all about \blue{encrypting} and \blue{decrypting} data using \blue{security keys}.
  \end{block}
  \begin{block}{Notation}
    \mathexpr{K(\id{data})} denotes that we \blue{use key \mathexpr{K}} to \blue{encrypt/decrypt \id{data}}.
  \end{block}
\end{slide}
\begin{slide}{Security mechanisms}
  \begin{block}{Symmetric cryptosystem}
    With \blue{encryption key \mathexpr{E_K(\id{data})}} and \blue{decryption key \mathexpr{D_K(\id{data})}}:
    \(
    \mbox{\itshape if\ } \id{data} = D_K(E_K(\id{data})) \mbox{\itshape \ then\ } \id{D_K} = \id{E_K}.
    \)
    Note: encryption and descryption key are the same and should be kept \red{secret}.
  \end{block}
  \begin{block}{Asymmetric cryptosystem}
    Distinguish a \red{public key} \blue{\mathexpr{PK(\id{data})}} and a \red{private} (\blue{secret})
    \red{key} \blue{\mathexpr{SK(\id{data})}}.
    \begin{itemize}\tightlist
    \item Encrypt message from \id{Alice} to \id{Bob}:
      \(
      \id{data} =
      \underbrace{%
        \red{SK_{bob}(}\overbrace{%
          \blue{PK_{bob}(\id{data})}
        }^{\mbox{\tiny Sent by Alice}}\red{)}
      }_{\mbox{\tiny Action by Bob}}
      \)
    \item Sign message for \id{Bob} by \id{Alice}:
      \(
        [\id{data}, \underbrace{%
            \red{\id{data}\stackrel{?}{=}PK_{alice}(}\blue{SK_{alice}(\id{data})}\red{)}
          }_{\mbox{\tiny Check by Bob}}
        ] =
        \underbrace{%
          [\id{data}, \blue{SK_{alice}(\id{data})}]
        }_{\mbox{\tiny Sent by Alice}}
      \)
    \end{itemize}
  \end{block}
\end{slide}
\begin{slide}{Security mechanisms}
  \begin{block}{Secure hashing}
    In practice, we use \blue{secure hash functions}: \mathexpr{H(\id{data})} returns a \red{fixed-length
      string}. \vspace*{-6pt}
    \begin{itemize}\tightlist
    \item \blue{Any change} from \id{data} to \id{data^*} will lead to a \blue{completely different string}
      \mathexpr{H(\id{data^*})}.
    \item Given a hash value, it is computationally impossible to find a \id{data} with \mathexpr{h = H(\id{data})}
    \end{itemize}
  \end{block}
  \onslide<2->
  \begin{block}{Practical digital signatures}
    Sign message for \id{Bob} by \id{Alice}:
      \[
        [\id{data}, \underbrace{%
            \red{H(data)\stackrel{?}{=}PK_{alice}(}\blue{sgn}\red{)}
          }_{\mbox{\tiny Check by Bob}}
        ] =
        \underbrace{%
          [\id{data}, H, \blue{\id{sgn} = SK_{alice}(H(\id{data}))}]
        }_{\mbox{\tiny Sent by Alice}}
      \]
  \end{block}

\end{slide}
\subsection{Scalability}
\begin{slide}{Scale in distributed systems}
  \begin{block}{Observation}
    Many developers  of modern distributed systems easily  use the adjective ``scalable''  without making clear
    \red{why} their system actually scales.
  \end{block}
  \onslide<2->
  \begin{block}{At least three components}
    \begin{itemize}
    \item Number of users or processes \red{(size scalability)}
    \item Maximum distance between nodes \red{(geographical scalability)}
    \item Number of administrative domains \red{(administrative scalability)}
    \end{itemize}
  \end{block}
  \onslide<3->
  \begin{alertblock}{Observation}
    Most systems account only, to a certain extent, for size scalability. Often a solution: multiple powerful
    servers operating independently in parallel. Today, the challenge still lies in geographical and
    administrative scalability.
  \end{alertblock}
\end{slide}
\begin{slide}{Size scalability}
  \begin{block}{Root causes for scalability problems with centralized solutions}
    \begin{itemize}
    \item The computational capacity, limited by the CPUs
    \item The storage capacity, including the transfer rate between CPUs and disks
    \item The network between the user and the centralized service
    \end{itemize}
  \end{block}
\end{slide}
  \begin{slide}{Formal analysis}
    \begin{block}{A centralized service can be modeled as a simple queuing system}
      \centering\includefigure{01-05}
    \end{block}
    \begin{block}{Assumptions and notations}
      \begin{itemize}\tightlist
      \item The queue has infinite capacity \mathexpr{\Rightarrow} arrival rate of requests is not influenced
        by current queue length or what is being processed.
      \item Arrival rate requests: \mathexpr{\lambda}
      \item Processing capacity service: \mathexpr{\mu} requests per second
      \end{itemize}
    \end{block}
    \begin{alertblock}{Fraction of time having \mathexpr{k} requests in the system}
      \[ p_k = \bigl(1 - \frac{\lambda}{\mu}\bigr)\bigl(\frac{\lambda}{\mu}\bigr)^k \]
    \end{alertblock}
  \end{slide}
  \begin{slide}{Formal analysis}
    \begin{block}{Utilization \mathexpr{U} of a service is the fraction of time that it is busy}
      \[ U = \sum_{k > 0} p_k = 1 - p_0 = \frac{\lambda}{\mu} \Rightarrow p_k = (1-U) U^k \]
    \end{block}
    \begin{block}{Average number of requests in the system}
      \[ \average{N} = \sum_{k\geq 0} k \cdot p_k = \sum_{k \geq 0} k \cdot (1-U)U^k = (1-U)\sum_{k\geq 0} k\cdot U^k
      = \frac{(1-U)U}{(1-U)^2} = \frac{U}{1-U} \]
    \end{block}
    \begin{block}{Average throughput}
      \[ X = \underbrace{U \cdot \mu}_{\mbox{\tiny server at work}} + \underbrace{(1-U) \cdot 0}_{\mbox{\tiny
        server idle}} = \frac{\lambda}{\mu} \cdot \mu = \lambda \]
    \end{block}
  \end{slide}
  \begin{slide}{Formal analysis}
    \begin{block}{Response time: total time take to process a request after submission}
      \[ R = \frac{\average{N}}{X} = \frac{S}{1-U} \Rightarrow \frac{R}{S} = \frac{1}{1-U} \]
    with \mathexpr{S = \frac{1}{\mu}} being the service time.
    \end{block}
    \begin{block}{Observations}
      \begin{itemize}
      \item If \mathexpr{U} is small, response-to-service time is close to 1: a request is immediately
        processed
      \item If \mathexpr{U} goes up to 1, the system comes to a grinding halt. \newline Solution: decrease \mathexpr{S}.
      \end{itemize}
    \end{block}
  \end{slide}
\begin{slide}{Problems with geographical scalability}
  \begin{block}{}
    \begin{itemize}
    \item Cannot simply go from LAN to WAN: many distributed systems assume \red{synchronous client-server
      interactions}: client sends request and waits for an answer. \blue{Latency} may easily prohibit this
      scheme.
    \item WAN links are often inherently \blue{unreliable}: simply moving streaming video from LAN to WAN is
      bound to fail.
    \item \blue{Lack of multipoint communication}, so that a simple search broadcast cannot be
      deployed. Solution is to develop separate \red{naming} and \red{directory services} (having their own
      scalability problems).
    \end{itemize}
  \end{block}
\end{slide}
\begin{slide}{Problems with administrative scalability}
  \begin{block}{Essence}
    Conflicting policies concerning usage (and thus payment), management, and security
  \end{block}

  \begin{exampleblock}{Examples}
    \begin{itemize}\firmlist
    \item \red{Computational grids}: share expensive resources between different domains. 
    \item \red{Shared equipment}: how to control, manage, and use a shared radio telescope constructed as
      large-scale shared sensor network?
    \end{itemize}
  \end{exampleblock}

  \begin{alertblock}{Exception: several peer-to-peer networks}
    \begin{itemize}\tightlist
    \item File-sharing systems (based, e.g., on BitTorrent)
    \item Peer-to-peer telephony (early versions of Skype)
    \item Peer-assisted audio streaming (Spotify)
    \end{itemize}\vspace*{-6pt}
    Note: \blue{end users} collaborate and not \blue{administrative entities}.
  \end{alertblock}
\end{slide}
\begin{slide}{Techniques for scaling}
  \begin{block}{Hide communication latencies}
    \begin{itemize}
    \item Make use of \blue{asynchronous communication}
    \item Have separate handler for incoming response
    \item \blue{Problem:} not every application fits this model
    \end{itemize}
  \end{block}
\end{slide}
\begin{slide}{Techniques for scaling}
  \begin{block}{Facilitate solution by moving computations to client}
    \begin{center}
      \includefigure{01-06a} \newline
      \includefigure{01-06b} \newline
    \end{center}
  \end{block}
\end{slide}
\begin{slide}{Techniques for scaling}
  \begin{block}{Partition data and computations across multiple machines}
    \begin{itemize}
    \item Move computations to clients (Java applets and scripts)
    \item Decentralized naming services (DNS)
    \item Decentralized information systems (WWW)
    \end{itemize}
  \end{block}
\end{slide}
\begin{slide}{Techniques for scaling} 
  \begin{block}{Replication and caching: Make copies of data available at different machines}
    \begin{itemize}
    \item Replicated file servers and databases
    \item Mirrored Websites
    \item Web caches (in browsers and proxies)
    \item File caching (at server and client)
    \end{itemize}
  \end{block}
\end{slide}
\begin{slide}{Scaling: The problem with replication}
  \begin{block}{Applying replication is easy, except for one thing}
    \begin{itemize}
	\item<2-> Having multiple copies (cached or replicated), leads to \red{inconsistencies}: modifying one copy
      makes that copy different from the rest.
	\item<3-> Always keeping copies consistent and in a general way requires \red{global synchronization} on
      each modification.
	\item<4-> Global synchronization precludes large-scale solutions.
    \end{itemize}
  \end{block}
  \onslide<5->
  \begin{block}{Observation}
    If we can tolerate inconsistencies, we may reduce the need for global synchronization, but    
    \blue{tolerating inconsistencies is application dependent}.
  \end{block}
\end{slide}
\section{A simple classification of distributed systems}
\subsection{High-performance distributed computing}
  \begin{slide}{Parallel computing}
    \begin{block}{Observation}
      High-performance distributed computing started with parallel computing
    \end{block}
    \begin{block}{Multiprocessor and multicore versus multicomputer}
      \begin{tabular}{cc}
        \includefigure{01-08a} &
        \includefigure{01-08b} 
      \end{tabular}
    \end{block}
  \end{slide}
  \begin{slide}{Distributed shared memory systems}
    \begin{block}{Observation}
      Multiprocessors are relatively easy to program in comparison to multicomputers, yet have problems when
      increasing the number of processors (or cores). \blue{Solution}: Try to implement a \red{shared-memory
        model} on top of a multicomputer.
    \end{block}
    \begin{exampleblock}{Example through virtual-memory techniques}
      Map all main-memory pages (from different processors) into one \blue{single virtual address space}. If
      a process at processor \id{A} addresses a page \id{P} located at processor \id{B}, the OS at \id{A}
      \blue{traps and fetches \id{P}} from \id{B}, just as it would if \id{P} had been located on local disk.
    \end{exampleblock}
    \begin{block}{Problem}
      Performance of distributed shared memory could never compete with that of multiprocessors, and failed to
      meet the expectations of programmers. It has been widely abandoned by now.
    \end{block}
  \end{slide}
\begin{slide}{Cluster computing}
  \begin{block}{Essentially a group of high-end systems connected through a LAN}
    \begin{itemize}\tightlist
    \item Homogeneous: same OS, near-identical hardware
    \item Single, or tightly coupled managing node(s)
    \end{itemize}
  \end{block}
  \begin{center}
    \includefigure{01-09}
  \end{center}
\end{slide}
\begin{slide}{Grid computing}

  \begin{block}{The next step: plenty of nodes from everywhere}
    \begin{itemize}
    \item Heterogeneous
    \item Dispersed across several organizations
    \item Can easily span a wide-area network
    \end{itemize}
  \end{block}
  \begin{alertblock}{Note}
    To allow for collaborations, grids generally use \red{virtual organizations}. In essence, this is a
    grouping of users (or better: their IDs) that allows for authorization on resource allocation.
  \end{alertblock}
\end{slide}
\begin{slide}{Architecture for grid computing}
  \begin{tabular}{@{}cc}
    \begin{minipage}{0.4\textwidth}
      \begin{center}
        \includefigure{01-10}
      \end{center}
    \end{minipage} &
    \begin{minipage}{0.55\textwidth}
  \begin{block}{The layers}\footnotesize
    \begin{itemize}\firmlist
    \item \blue{Fabric}: Provides interfaces to local resources (for querying state and capabilities, locking,
      etc.)
    \item \blue{Connectivity}: Communication/transaction protocols, e.g., for moving data between
      resources. Also various authentication protocols.
    \item \blue{Resource}: Manages a single resource, such as creating processes or reading data.
    \item \blue{Collective}: Handles access to multiple resources: discovery, scheduling, replication.
    \item \blue{Application}: Contains actual grid applications in a single organization.
    \end{itemize}
  \end{block}
    \end{minipage}
  \end{tabular}
\end{slide}
\subsection{Distributed information systems}
\begin{slide}{Integrating applications}

  \begin{block}{Situation}
    Organizations confronted with many \blue{networked applications}, but achieving interoperability was
    painful.
  \end{block}

  \begin{block}{Basic approach}
    A networked application is one that runs on a \red{server} making its services available to remote
    \red{clients}. Simple integration: clients combine requests for (different) applications; send that off;
    collect responses, and present a coherent result to the user.
  \end{block}

  \begin{block}{Next step}
    Allow direct application-to-application communication, leading to \red{Enterprise Application
      Integration}.
  \end{block}

\end{slide}
\begin{slide}{Example EAI: (nested) transactions}
  \vspace*{-6pt}
  \begin{block}{Transaction}
    \begin{center}
      \sffamily\footnotesize
      \renewcommand{\arraystretch}{1.1}
      \begin{tabular}{|l|l|} \hline
        \textbf{Primitive} & \textbf{Description} \\ \whline
        \idsn{BEGIN\_TRANSACTION} & Mark the start of a transaction  \\ \hline
        \id{END\_TRANSACTION} & Terminate the transaction and try to commit  \\ \hline
        \id{ABORT\_TRANSACTION} & Kill the transaction and restore the old values  \\ \hline
        \id{READ} & Read data from a file, a table, or otherwise  \\ \hline
        \id{WRITE} & Write data to a file, a table, or otherwise  \\ \hline
      \end{tabular}
    \end{center}
  \end{block}
  \begin{block}{Issue: all-or-nothing}
    \begin{columns}[T]
      \begin{column}{0.3\textwidth}
        \includefigure[0.7]{01-12}
      \end{column}
      \begin{column}{0.65\textwidth}\footnotesize
        \begin{itemize}\tightlist
        \item \blue{Atomic}: happens indivisibly (seemingly)
        \item \blue{Consistent}: does not violate system invariants
        \item \blue{Isolated}: not mutual interference
        \item \blue{Durable}: commit means changes are permanent
        \end{itemize}
      \end{column}
    \end{columns}
  \end{block}
\end{slide}
\begin{slide}{TPM: Transaction Processing Monitor}
  \begin{centerfig}
    \includefigure{01-13}
  \end{centerfig}
  \begin{block}{Observation} 
    Often, the data involved in a transaction is distributed across several servers. A \red{TP
      Monitor} is responsible for coordinating the execution of a transaction.
  \end{block}
\end{slide}
\begin{slide}{Middleware and EAI}
  \begin{centerfig}
    \includefigure{01-14}
  \end{centerfig}
  \begin{block}{Middleware offers communication facilities for integration}
    \begin{description}[xxx]
      \item[\blue{Remote Procedure Call (RPC)}:] Requests are sent through local procedure call,
        packaged as message, processed, responded through message, and result returned as return from call.
      \item[\blue{Message Oriented Middleware (MOM)}:] Messages are sent to logical contact point (\blue{published}),
        and forwarded to \blue{subscribed} applications.
    \end{description}
  \end{block}
\end{slide}
  \begin{slide}{How to integrate applications}
    \begin{block}{}
      \begin{description}[xxx]

      \item[\blue{File transfer}:] Technically simple, but not flexible:
        \begin{itemize}\firmlist
        \item Figure out file format and layout
        \item Figure out file management
        \item Update propagation, and update notifications.
        \end{itemize}

      \item[\blue{Shared database}:] Much more flexible, but still requires common data scheme next to risk of
        bottleneck.

      \item[\blue{Remote procedure call}:] Effective when execution of a series of actions is needed.

      \item[\blue{Messaging}:] RPCs require caller and callee to be up and running at the same time. Messaging allows
        decoupling in time and space.

      \end{description}
    \end{block}
  \end{slide}
\subsection{Pervasive systems}
\begin{slide}{Distributed pervasive systems}
  \begin{block}{Observation} 
    Emerging next-generation of distributed systems in which nodes are small, mobile, and often embedded in a
    larger system, characterized by the fact that the system \red{naturally blends into the user's environment}.
  \end{block}
  \begin{block}{Three (overlapping) subtypes}
    \begin{itemize}
    \item<2-> \blue{Ubiquitous computing systems}: pervasive and \red{continuously present}, i.e., there is a
      continuous interaction between system and user.
    \item<3-> \blue{Mobile computing systems}: pervasive, but emphasis is on the fact that devices are
      \red{inherently mobile}. 
    \item<4-> \blue{Sensor (and actuator) networks}: pervasive, with emphasis on the actual (collaborative)
      \red{sensing} and \red{actuation} of the environment.
    \end{itemize}
  \end{block}
\end{slide}
\begin{slide}{Ubiquitous systems}
  \begin{block}{Core elements}
    \begin{enumerate}\tightlist
    \item (\blue{Distribution}) Devices are networked, distributed, and accessible transparently
    \item (\blue{Interaction}) Interaction between users and devices is highly unobtrusive
    \item (\blue{Context awareness}) The system is aware of a user's context to optimize interaction
    \item (\blue{Autonomy}) Devices operate autonomously without human intervention, and are thus highly self-managed
    \item (\blue{Intelligence}) The system as a whole can handle a wide range of dynamic actions and interactions
    \end{enumerate}
  \end{block}
\end{slide}
\begin{slide}{Mobile computing}
  \begin{block}{Distinctive features}
    \begin{itemize}
    \item A myriad of different mobile devices (smartphones, tablets, GPS devices, remote controls, active
      badges).
    \item Mobile implies that a device's location is expected to change over time \mathexpr{\Rightarrow}
      change of local services, reachability, etc. Keyword: \red{discovery}.
    \item Maintaining stable communication can introduce serious problems.
    \item For a long time, research has focused on directly sharing resources between mobile devices. It never
      became popular and is by now considered to be a fruitless path for research.
    \end{itemize}
  \end{block}
  \onslide<2->
  \begin{block}{Bottomline}
    Mobile devices set up connections to stationary servers, essentially bringing mobile computing in the
    position of clients of cloud-based services.
  \end{block}
\end{slide}
\begin{slide}{Mobile computing}
  \footnotesize
  \begin{centerfig}
    \includefigure{01-15a} \\
    Mobile cloud computing \\
    \ \\
    \includefigure{01-15b} \\
    Mobile edge computing
  \end{centerfig}
\end{slide}
\begin{slide}{Sensor networks}
  \begin{block}{Characteristics}
    The \blue{nodes} to which sensors are attached are:
    \begin{itemize}
    \item Many (10s-1000s)
    \item Simple (small memory/compute/communication capacity)
    \item Often battery-powered (or even battery-less)
    \end{itemize}
  \end{block}
\end{slide}
\begin{slide}{Sensor networks as distributed databases}
  \begin{block}{Two extremes}
    \begin{center}
      \begin{tabular}{c}
        \includefigure{01-16a} \\
        \includefigure{01-16b} \\
      \end{tabular}
    \end{center}
  \end{block}
\end{slide}
\begin{slide}{The cloud-edge continuum}
  \begin{centerfig}
    \includefigure{01-17}
  \end{centerfig}
\end{slide}
\section{Pitfalls}
\begin{slide}{Developing distributed systems: Pitfalls}
  \begin{block}{Observation}
    Many distributed systems are needlessly complex, caused by mistakes that required patching later on. 
    Many \blue{false assumptions} are often made.
  \end{block}
  \onslide<2->
  \begin{block}{False (and often hidden) assumptions}
    \begin{itemize}
    \item<3-> The network is reliable
    \item<4-> The network is secure
    \item<5-> The network is homogeneous
    \item<6-> The topology does not change
    \item<7-> Latency is zero
    \item<8-> Bandwidth is infinite
    \item<9-> Transport cost is zero
    \item<10-> There is one administrator
    \end{itemize}
  \end{block}
\end{slide}
\section{Summary}
